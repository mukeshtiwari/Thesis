\chapter{Homomorphic Schulze Algorithm : Axiomatic Approach}
\label{cha:homormorphic_schulze}
Same as the last chapter, introduce the motivation and the high-level picture to
readers, and introduce the sections in this chapter. 

The problem with the previous methods is it's clearly exposes the ballots 
in plaintext possible leading to coercion. 

\section{Verifiable Homomorphic Tallying}
  \subsection{Ballot Representation}
     Argue here why do we changed the ballot representation 
     to encrypted matrix
     \subsubsection{Validity of Ballots}
  \subsection{Cryptographic Primitives}
  	\subsubsection{Construction Primitive}
  	  Write here about how you construct the data
  	\subsubsection{Verification Primitive}
  	  Write here about how you verify the constructed 
  	  data

\section{Realization in Theorem Prover}
    Discuss the primitives and axioms
\section{Correctness by Construction}
  Describe inductive data type and assertion in the 
  system to make sure you can not construct any 
  illegal term
\section{Implementation : A Experience Report}
  \subsection{Extraction}
  \subsection{Unicrypt : Java Library for Realizing Cryptographic Primitives}
     
\section{Experimental Result and Scrutiny Sheet} 
  Here goes the result and certificate
  Also point here the weakness of the system is
\section{summary}
  Discuss here the weakness of system and pave the path to next chapter
  that how software independence can help in verifying the election