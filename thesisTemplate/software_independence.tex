\chapter{Scrutiny Sheet : Software Independence}
\label{cha:software_independence}

\epigraph{Somewhere inside of all of us is the power to change the world.} 
{\textit{Roald Dahl }}

Getting everything right in electronic voting is very difficult, and assuming, for a moment,
that everything is correct, convincing this fact to every stakeholder and  any independent third party 
is almost next to impossible.  As we stated in earlier chapter [Give a link to the section] that 
software is complex artefact and, often, poorly design and tested. It should not come as a surprise to anyone
that we are far more competent in producing the incorrect software than producing a provable correct one. 
 In order to tackle the software complexity problem and ensure the 
public trust in process, Ronald Rivest  proposed "software-independence". 

\textit{A voting system is software independent if an undetected change or error
   in its software can not cause an undetected change or error in an 
   election outcome.}
 
 \noindent
 Software-independence put forward the 
 In his paper
 Software independence is one form of auditability, enabling detection and 
   possible correction of election outcome errors caused by malicious software or software
bugs.



Same as the last chapter, introduce the motivation and the high-level picture to
readers, and introduce the sections in this chapter. 

One of the crucial 
aspect in electronic voting is ability to detect change in outcome of 
election because of software bugs. 


   A voting system is software independent if an undetected change or error
   in its software can not cause an undetected change or error in an 
   election outcome.
%\section{Verification and Verifiability}
%  Paste Evote section 2

\section{Certificate : Ingredient for Verification}
  \fix{There are two notion of verification. Software verification and 
   election verification. A electronic voting scheme implemented in 
   Coq is verified implementation, but it does not imply that method 
   itself is verifiable. Explain the software independence } 
   
   \subsection{Plaintext Ballot Certificate}
    Flesh out the details needed here for writing proof checker
    
    \subsection{Encrypted Ballot Certificate}
    Flesh out the details needed here for writing proof checker
    
   
\section{Proof Checker}
  Write the details of proof checker for both certificates
  
  	\subsection{A Verified Proof Checker : IACR 2018}
  	Write some details about proof checker.
  
\section{Summary}
   Write some advantage of proof checker for certificates.
   To create the mass scrutineers, all we need is a simple proof checker
   which would take proof certificate as input and spit true or false.
   If it's true then we accept the outcome of election otherwise something 
   wrong.