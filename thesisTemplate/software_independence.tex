\chapter{Scrutiny Sheet : Software Independence}
\label{cha:software_independence}
Same as the last chapter, introduce the motivation and the high-level picture to
readers, and introduce the sections in this chapter. 

One of the crucial 
aspect in electronic voting is ability to detect change in outcome of 
election because of software bugs. 


   A voting system is software independent if an undetected change or error
   in its software can not cause an undetected change or error in an 
   election outcome.
\section{Verification and Verifiability}
  Paste Evote section 2

\section{Certificate : Ingredient for Verification}
  \fix{There are two notion of verification. Software verification and 
   election verification. A electronic voting scheme implemented in 
   Coq is verified implementation, but it does not imply that method 
   itself is verifiable. Explain the software independence } 
   
   \subsection{Plaintext Ballot Certificate}
    Flesh out the details needed here for writing proof checker
    
    \subsection{Encrypted Ballot Certificate}
    Flesh out the details needed here for writing proof checker
    
   
\section{Proof Checker}
  Write the details of proof checker for both certificates
  
  	\subsection{A Verified Proof Checker : IACR 2018}
  	Write some details about proof checker.
  
\section{Summary}
   Write some advantage of proof checker for certificates.
   To create the mass scrutineers, all we need is a simple proof checker
   which would take proof certificate as input and spit true or false.
   If it's true then we accept the outcome of election otherwise something 
   wrong.