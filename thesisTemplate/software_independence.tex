\chapter{Scrutiny Sheet : Software Independence}
\label{cha:software_independence}
\setlength{\parindent}{2em}

\epigraph{Somewhere inside of all of us is the power to change the world.} 
{\textit{Roald Dahl }}

\section{Introduction}


One of the major disadvantage of using cryptography 
to achieve privacy (using encryption to make the 
content of ballot private)  and verifiability (using zero-knowledge-proof for verification of claims) makes the verification 
process cumbersome. As a consequence, the verification process (checking the scrutiny sheet) is only viable for
tiny fraction of general population, cryptographers, results into a sharp decrease in number of scrutineers. 
While it is not very difficult to find cryptographers to verify the election, 
they are, off course, not the representative population in any democracy. 
In order to increase the number of scrutineers and subsequently confidence in electronic voting, we follow the 
route of providing a formally verified open-source 
reference certificate checker which anyone can inspect and run on the election data. 
  The rationale behind formally verifying the certificate checker is \emph{correctness}
  and open sourcing is to gain the public trust  via careful examination.  
  For example, consider a scenario where we do not provide the reference checker,
  then how 
  likely would it be for community/voters to develop the 
  verified checker? Moreover, assuming that we publish one unverified certificate checker,
  what would happen if it returns false on a valid certificate because of its own bug? 
  Both situations, of course, would be a devastating situation, so not only 
  should we provide a reference certificate checker, but it should be a formally verified one. 
  Additionally, a formally verified reference certificate checker would open the gate for
  debate in case of someone's implementation for checking certificate diverges from the reference checker.
In the case of a diverging situation, there are two possibility, either the reference checker is verified 
using wrong assumptions,  or the implementation itself is wrong.  The first situation is certainly 
not very pleasant because it would deteriorate the public trust in the system, but nonetheless, it is always
good to  have openness in democracy to make it more strong. 
 


In this chapter, we discuss the concepts required to develop a verified certificate checker for the certificate we generated 
in the last chapter. Moreover, we sketch pseudo code (and pen-and-paper proof) as well for these concepts 
which can be easily translated into the Coq code.  We have already explained our certificate in the Section \ref{sec:extract}, 
but intuitively,  checking our certificate amounts to proving that  the homomorphic margin has been 
computed correctly and zero-knowledge-proof for every claim is correct. In a nutshell, 
our claims were:
\begin{enumerate}
\item honest decryption zero-knowledge-proof: every encrypted value is decrypted honestly 
\item shuffle zero-knowledge-proof:
  a ballot has been permuted by the same permutation whose commitment is published (commitment consistent shuffle \citep{Wikstrom:2009:CPS}).
\item final homomorphic tally is computed correctly
\end{enumerate}

\noindent
We sketch the encoding of Pedersen's commitment, one of the primitives of shuffle zero-knowledge-proof, in Coq, 
but we leave the other details of shuffle zero-knowledge-proof algorithm \citep{Wikstrom:2009:CPS}. Moreover, we
encode the sigma protocol in Coq as a record and give various examples of concrete sigma protocol including 
the honest decryption zero-knowledge-proof.  Finally, all 
the concepts explained in this chapter are sufficient to develop the formally verified 
certificate checker for IACR election 
\footnote{https://vote.heliosvoting.org/helios/elections/60a714ea-ce6d-11e8-8248-76b4ab96574c/view} 
and our proof development can be accessed 
\footnote{https://github.com/mukeshtiwari/secure-e-voting-with-coq}.


\textbf{Chapter Outline:} In the Section $\ref{sec:algebra}$, we discuss the underlying algebraic structures needed for various 
 cryptographic operations. Section $\ref{sec:pedersen}$ focuses on generalizing the Pedersen commitment scheme for 
 a matrix. In the following Section $\ref{sec:sigma_coq}$, we discuss the details of sigma protocol, and its formalization 
 in Coq. In order to eschew the (monadic) probabilistic reasoning of sigma protocols, we use the standard trick of making the randomness 
 explicit to make the reasoning easier
 without losing the meaning of sigma protocol. In the Section $\ref{sec:conc_sigma}$, we show 
 how to make a concrete instance of sigma protocol by giving an example of discrete logarithm. In addition, 
 we show the protocol needed for honest decryption (Section $\ref{sec:dec_sigma}$). Section $\ref{sec:homo_tally}$ sketches 
 the homomorphically tally based on additive ElGamal scheme. Finally, we discuss the IACR scrutiny sheet in the Section $\ref{sec:election_iacr}$, 
 and we summarize the chapter in the Section $\ref{sec:summary_software}$


\section{Algebraic Structures: Building Blocks}
\label{sec:algebra}
The basic building blocks any cryptographic system are algebraic structures, specifically cyclic group of prime order, field and vector space. In general, 
we do not need vector space, and group and field are sufficient for most of the cryptographic purposes. However,
vector space  of a cyclic group of prime order over the field of integers modulo the same order is nicer to work 
because multiplying a group element by a field element can be abstracted over scalar multiplication of vector space.

\begin{definition}[Group] 
A group is a set $G$, with a binary operator $\cdot : G \rightarrow G \rightarrow G$, identity element $e$, and inverse operator $inv : G \rightarrow G$ such 
    that the following laws hold:  \end{definition} 
    \begin{itemize}
     \item \textit{Associativity}: $\forall$  a b c $\in G,$  $a \cdot (b \cdot c) = (a \cdot b) \cdot c$
    \item \textit{Closure}: $\forall$ a b $\in G,$  $a \cdot b \in G$
    \item \textit{Inverse Element}: $\forall$ a $\in G$, $a \cdot (inv \text{ } a) = (inv \text{ } a) \cdot a = e$
    \item \textit{Identity}: $\forall$ a $\in G,$  $a \cdot e = e \cdot a  = a$
    \end{itemize}
  
    \noindent
    If the group is commutative, i.e. $\forall$ a b $\in  G,$  $a \cdot b = b \cdot a$, then we call it Abelian group.  We can represent the Abelian group in Coq by using the 
    record data type. 
 
 \begin{minted}{Coq}

Record AbelGroup (G : Type) 
    (Hdec : forall x y : G, {x = y} + {x <> y}) 
    (dot : G -> G -> G)  (inv : G -> G) (e : G) :=
  {
    dot_associativity : forall x y z, 
      dot x (dot y z) = dot (dot x y) z;
    dot_left : forall x, dot x e = x;
    dot_right : forall x, dot e x = x;
    left_inverse : forall x, dot (inv x) x = e;
    right_inverse : forall x, dot x (inv x) = e;
    commutative : forall x y, dot x y = dot y x
  }.
  
\end{minted}

\noindent
Our Coq encoding is slight different from the definition of the group we gave above, mainly that \textit{G : Type} "read it as A has the type Type" and 
\textit{(Hdec : forall x y : G, {x = y} + {x <> y})}. Because of the theoretical foundations of Coq, every term in Coq has to have a type;
hence we need to explicitly state the type of term $G$, and we assume a decidable equality $Hdec$ on the elements of $G$.  
In a nutshell, Type is used to represent sets. 

\begin{definition}[Field] 
A field  is a set $\mathbb{F}$, with two binary operator $+ : \mathbb{F} \rightarrow \mathbb{F} \rightarrow \mathbb{F}$,  and $\cdot : \mathbb{F} \rightarrow \mathbb{F} \rightarrow \mathbb{F}$, 
two identity element $0$ and $1$, and two unary operator $- : \mathbb{F} \rightarrow \mathbb{F}$, $1/ : \mathbb{F} \rightarrow \mathbb{F}$  such that:
\end{definition} 
 \begin{itemize}
 \item ($\mathbb{F}$, +, 0, -) forms an abelian group.
 \item ($\mathbb{F}$ - $\lbrace 0 \rbrace$, $\cdot$, 1, 1/) forms an abelian group.
 \item $\cdot$ distributes over +.
 \end{itemize}
 
 

\begin{definition}[Vector Space]
A set $V$ with two binary operation $+ : V \rightarrow V \rightarrow V$ and $\cdot : \mathbb{F}  \rightarrow V \rightarrow V$, 
is  a vector space over a field $\mathbb{F}$ if the following properties hold:
\end{definition} 
\begin{itemize}
 \item Closure under addition: (V, +) forms an abelian group. 
 \item Scalar  distribute  with respect  to  vector  addition: $\forall$  $r \in \mathbb{F}$, $u$ $v$ $\in V$,  $r \cdot (u + v) = r \cdot u + r \cdot v$
 \item Scalar distribute with respect to field addition: 
                $\forall$  $a$ $b$ $\in \mathbb{F}$, $u$ $\in V$, $(a +_{\mathbb{F}} b) \cdot u = a \cdot u + b \cdot v$
  \item Scalar multiplication is associative with respect to field multiplication:
         $\forall$ $a$ $b$ $\in  \mathbb{F}$, $u$ $\in V$ , $(a \cdot_{\mathbb{F}} b) \cdot u = a \cdot (b \cdot u)$
  \item Identity: $\forall a \in V, 1_{\mathbb{F}} \cdot a = a$

\end{itemize}



\section{Pedersen Commitment Scheme}
\label{sec:pedersen}
Recall that in the last chapter to prove that if a ballot was valid or invalid, we generated a secret permutation $\pi$ and published its commitment using 
Pedersen commitment scheme. Later, we use this to shuffle each row and column of the ballot by this (secret) permutation. 
In the shuffle algorithm \citep{Wikstrom:2009:CPS}), the data structure of permutation $\pi$ is matrix (a permutation matrix to be precise).
In this section, we discuss that how to use generalize Pedersen commitment scheme for matrix data structure. 

A Pedersen commitment for any two give group element $g$ $h$ $\in G$, a message $m \in \mathbb{F}$, and a random element 
$r \in \mathbb{F}$ is $g^r \cdot h^m$ ($\cdot$ is the group operation). In Coq, it can be encoded as follows:  


\begin{minted}{coq}

Definition ped_commitment {F G} (fn : G -> F -> G) 
  (gn : G -> G -> G) (g h : G) (r m : F) : G := 
    gn (fn g r) (fn h m).


\end{minted}

\noindent
The function $fn$ (exponentiation operator) can be abstracted as a  scalar binary operator $\cdot$ and the 
function $gn$ (group operator) can be 
abstracted as addtion binary operator $+$ of vector space (vector space of a cyclic group $G$ of prime 
order over the field of integers $\mathbb{F}$ modulo the same order).  
Even though this abstraction is not necessary, but it 
provides a clean way to work with a \textit{Group} and \textit{Field}. 
We can extend the Pedersen commitment for a group element $g$, 
vector of $n$ group element $h_{1}, h_{2} \dots h_{n}$, vector of $n$ 
field element $m_{1}, m_{2} \dots m_{n}$, and a random field element 
$r$ is $g^r \cdot \prod_{i = 1}^n  h_{i}^{m_{i}}$. 
The $\prod_{i = 1}^n  h_{i}^{m_{i}}$ function can be written in Coq as 
(this program is written using Equation library \citep{Sozeau:2019:ERH:3352468.3341690}):

\begin{minted}{coq}


Equations prod {F G n}  (fn : G -> F -> G) (gn : G -> G -> G)
  (hi : Vector.t G (S n)) (mi : Vector.t F (S n)) :  G :=
  prod fn gn (vcons h vnil) (vcons m vnil) := fn h m;
  prod fn gn (vcons h hs)  (vcons m ms)  := 
         gn (fn h m) (prod fn gn hs ms).


\end{minted}


Now we can define  generalized Pedersen commitment ( $g^r \cdot  \prod_{i = 1}^n  h_{i}^{m_{i}}$) as:

\begin{minted}{coq}

Definition ped_gen_commitment {F G n}
           (fn : G -> F -> G) (gn : G -> G -> G)
           (g : G) (r : F) (hs : Vector.t G (S n))
           (ms : Vector.t F (S n)) :=
  gn (fn g r) (prod fn gn hs ms).
  
\end{minted}
  
 
Finally, to commit a matrix of size $N \times N$, we need to call 
$ped\_gen\_commitment$ on each column of matrix. Consequently, 
we  would get a vector of commitments of size N.

\section{Sigma Protocol: Efficient Zero-Knowledge-Proof}
\label{sec:sigma_coq}
A sigma protocol is a two party protocol, a prover $P$ and a verified $V$, where prover $P$ tries to convince the verifier $V$ that she 
holds a private input $x$ for some public input $w$ such that a binary relation $R$ holds, i.e. $(x, w) \in R$.  Sigma protocol, 
in general, is a three step protocol:
\begin{enumerate}
\item Initialisation: $P$ generates a random message $r$, commits it, and send the committed message to $V$
\item Challenge: $V$ generates a random message $c$, and sends it to $V$
\item Response: $P$ sends a response $z$ to $V$
\end{enumerate} 

\noindent
Upon receiving the response $z$, $V$ either accepts the proof or rejects the proof.  Now we define the sigma protocol in 
Coq by using record data type.

\begin{minted}{coq}
Record SigmaProtocol (Statement : Type) (* Statement x *)
       (Witness : Type) (* witness w *)
       (Rel : Statement -> Witness -> bool) (* decidable relation *)
       (RandCoin : Type) (* random coin *) 
       (Commitment : Type) (* commitments *)
       (Challenge : Type) (* challenges *) 
       (Response : Type) (* response *) :=
  MkSigma {
  (* initial commitment send by the Prover *)
  initial : RandCoin -> Commitment;
  (* Randomness send by the verifier.  *) 
  challenge : Challenge;
  (* response generate by prover *)
  response : Statement -> Witness ->
     RandCoin -> Challenge ->
     Response;
  (* verify the response *)
  verify : Statement * Commitment * Challenge * Response
  -> bool;
  (* Simulator *)
  simulator : Statement -> Challenge -> Response ->
     Statement * Commitment * Challenge * Response;
  (* Extractor *)
  extractor : Challenge -> Response -> Challenge -> 
     Response -> Witness;
  

  (* Completness *)
  Completness : forall (s : Statement) (w : Witness)
    (r : RandCoin) (e : Challenge),vRel s w = true -> 
    verify (s, initial r, e, response s w r e) = true;

 (* Special Soundness *)
  Special_Soundness : forall s c e1 e2 r1 r2,
    e1 <> e2 ->
    verify (s, c, e1, r1) = true ->
    verify (s, c, e2, r2) = true ->
    Rel s (extractor e1 r1 e2 r2) = true;
  
 (* Special honest verifier zero knowledge proof. Explicit 
    randomness makes it nicer to work in theorem prover  *)
  Special_Honest_Verifier_ZKP  (s : Statement) 
    (w : Witness) (e : Challenge): 
    Rel s w = true -> forall (r : RandCoin), 
    verify (s, initial r, e, response s w r e) = true <->
    forall (z : Response), verify (simulator s e z) = true;
  
 (* simulator correct *)
  Simulator_correct : forall (s : Statement) 
   (e : Challenge) (r : Response),
   verify (simulator s e r) = true;
  }.
\end{minted}


\noindent
The record \textit{SigmaProtocol} is indexed by \textit{Statement}, the public input 
known to $P$ and $V$, \textit{Witness}, secret input known to $P$, 
\textit{Rel} such that $(x, w) \in Rel$, \textit{RandCoin}, the private 
random coin toss of $P$, \textit{Commitment}, commitment computed 
by $P$ based on the random coin toss of itself, 
\textit{Challenge}, the random challenge of $V$ to $P$, 
\textit{Response}, the response of $P$ send to $V$. Moreover, 
the body of record \textit{SigmaProtocol} contains
functions \textit{initial}, \textit{challenge}, \textit{response}, and \textit{verify}
to reflect the three steps of sigma protocol with two auxiliary 
functions \textit{simulator} and \textit{extractor}.  The 
reason for having \textit{simulator} and \textit{extractor} 
to prove the property of special honest verifier zero knowledge proof 
and special soundness of 
the sigma protocol. Finally, we have four properties \textit{Completeness}, 
\textit{Special\_Soundness}, \textit{Special\_Honest\_Verifier\_ZKP}
and \textit{Simulator\_correct}. 
\textit{Completeness} states that if $P$ and $V$ follow the protocol, then 
verifier would accept the proof. \textit{Special\_Soundness},
expresses that if $P$ is able to convince $V$ with two 
accepting transcript for the same commitment, then $V$ can extract the witness.
Recall that \textit{special honest 
verifier zero knowledge proof} amounts to a probabilistic polynomial time simulator 
$S$ which would generate a proof transcript for some statement 
$s$  with same probability distribution as if there were a real 
conversation between a prover $P$ and a verifier $V$ 
for the statement $s$ and witness $w$ such that $(s, w) \in  R$.
Informally, the real proof transcript depends on statement $s$, witness $w$,  
and challenge $e$, 
while the simulated proof transcript depends on statement $s$ and challenge $e$. 
(simulator does have not access to witness $w$, so to generate a accepting 
proof just by using $s$ and $e$, it uses a concept call rewinding.)
In our definition of \textit{Special\_Honest\_Verifier\_ZKP}, we eschew the 
probabilistic reasoning by making randomness explicit, and it
states that for any  given 
fixed statement $s$, witness $w$, challenge $e$ and assumption that $Rel$ $s$ $w$ holds, 
then for every random coin $r$  and a accepting real transcript, simulator can construct 
an accepting transcript from all random response drawn from response space. 
The \textit{Correct\_simulator} property states that the simulator is correct, i.e.
any transcript created by simulator checks out. 



%\noindent
%Our definition of sigma protocol is almost complete except we have not given 
%the special honest verifier zero knowledge proof axiom.  Recall that special honest 
%verifier zero knowledge proof amounts to a probabilistic polynomial time simulator 
%$S$ which would generate a proof transcript for some statement 
%$s$  with same probability distribution as if there were a real 
%conversation between a prover $P$ and a verifier $V$ 
%for the statement $s$ and witness $w$ such that $(s, w) \in  R$.
%Informally, the real proof transcript depends on statement $s$, witness $w$,  
%and challenge $e$, 
%while the simulated proof transcript depends on statement $s$ and challenge $e$. 
%(simulator does have not access to witness $w$, so to generate a accepting 
%proof just by using $s$ and $e$, it uses a concept call rewinding.)
%
%
%If we were doing probabilistic reasoning, we could have represent the real view and 
%simulator's view as following (not a real code, but inspired by Haskell):
%
%\begin{minted}{coq}
%Real_view (s : Statement) (w : Witness) (e : Challenge) := do
%  r <- G (* generate a random value from some group G *)
%  let a := g^r (* commitment *)
%  let z := r + e * w  (* compute the response *)
%  return (s, a, e, z)
%\end{minted}
%
%
%\begin{minted}{coq}
%Simulator_view (s : Statement) (e : Challenge) := do
% z <- G (* random element from some group G *)
% return (s, g^z * s^(-e), e, z)
%\end{minted}
%
%\begin{minted}{coq}
%
% Special_Honest_Verifier_ZKP  (s : Statement) (w : Witness) 
%  (e : Challenge) (H : Rel s w = true) := 
%  Pr [Real_view  s w e] = 
%  Pr [Simulator_view s e]
%
%\end{minted}
%
%\noindent
%Based on these two probabilistic view, we could have shown that two are equal. Intuitively, 
%in $Real\_view$, statement $s$, witness $w$, and the challenge $e$ is fixed, so the only thing that vary is the
%random coin drawn uniformly from some group $G$. The probability of drawing an element uniformly 
%from a given set $S$ is 1/|S| (| | represent the cardinality of a set), so the value of Pr [Real\_view  s w e] = 1/|G|. 
%By the same token of reasoning, Pr [Simulator\_view s e] = 1/|G| which concludes that 
%both probabilities are equal. However, the only problem is that we are not doing probabilistic reasoning, 
%but not all all hope is lost.
%One key observation to escape this probabilistic reasoning is 
%that we can make the randomness $r$ and response $z$ as 
%explicit parameter. Consequently, our probabilistic program would turn into 
%a deterministic Coq program. 
%
%
%\begin{minted}{coq}
%Real_view (s : Statement) (w : Witness) (e : Challenge) 
%  (r : RandCoin):=
%  let a := g^r in (* commitment *)
%  let z := r + e * w in  (* compute the response *)
%  (s, a, e, z)
%\end{minted}
%
%
%\begin{minted}{coq}
%Simulator_view (s : Statement) (e : Challenge) 
% (z : Response) :=
% (s, g^z s^(-e), e, z)
%
%\end{minted}
%
%
%
%Now that our views are deterministic, we need to find a way to model 
%the probability distribution of the two views. We solve this problem 
%by showing a bi-implication  between  the real view and simulated view.
%We show that real view and simulator's view align with each other. 
%In terms of Coq, it is expressed as:
%
%\begin{minted}{coq}
% (* Probablistic Reasoning could have made it nicer *)
%Special_Honest_Verifier_ZKP  (s : Statement) 
%(w : Witness) (e : Challenge): 
%Rel s w = true -> forall (r : RandCoin), 
%verify (s, initial r, e, response s w r e) = true <->
%forall (z : Response), verify (simulator s e z) = true;
%
%\end{minted}

%
%The axiom $Special\_Honest\_Verifier\_ZKP$ says that for any  given 
%fixed statement $s$, witness $w$, challenge $e$ and assumption that $Rel$ $s$ $w$ holds, 
%then for every random coin $r$  and a accepting real transcript, simulator can construct 
%an accepting transcript from all random response drawn from response space. 
%We add this axiom to the record data type $SigmaProtocol$ to make it 
%complete. 

Finally, we can use our construction, \textit{SigmaProtocol}, as building block 
for composing different sigma protocols, which we are not explaining 
here.  For example, 
we can define AND composition,
EQ composition, OR composition, etc.

\subsection{Concrete Sigma Protocol: Discrete Logarithm}
\label{sec:conc_sigma}
One of the most basic sigma protocol is proof of knowledge of 
discrete logarithm, i.e. given two elements $g$ and $h$ of 
a group $G$, prover convinces the verifier that 
she knows the discrete logarithm ($\log_g h$) in zero 
knowledge. In Camenisch-Stadler notation\citep{camenisch1997proof} of zero knowledge proof, it is represented as:
$ZKPoK \lbrace w \text{ | } h = g^w \rbrace$. We can show 
that this is a sigma protocol inside Coq by 
by encoding all the  functions 
and proving all the axioms mentioned in 
our record type $SigmaProtocol$.  For example,
 we can write the $initial$ function as taking a input 
random coin $r$ as input and computing 
$g^r$, $challenge$ as a function which simply returns 
a challenge $e$, and so forth: 


\begin{displayquote}

$\text{initial r := } g^r}$  

$\text{challenge := } e$

$\text{response h w r e := } r + e \cdot w$

$\text{verify h a e z  := } g^z = a \cdot h^e$

$\text{simulator s e z := } (g^z \cdot h^{-e}, e, z)$

$\text{extractor }  $c_{1}$ $z_{1}$ $c_{2}$ $z_{2}$ := (z_{1} - z_{2}) \cdot (c_{2} - c_{1})^{-1}$

\end{displayquote}

Based on these definitions, we can easily discharge the three proofs, \textit{Completeness}, 
\textit{Special Soundness}, and \textit{Special Honest Verifier Zero Knowledge} axiom by simple 
algebraic manipulation.

\subsection{Honest Decryption Zero Knowledge Proof}
\label{sec:dec_sigma}
We have sigma protocol at our arsenal, we focus on honest decryption 
problem. How can we convince someone that for a given group 
$(G, g, p, h)$ and private key $x$ ($h := g^x$), the message $m$ is the 
honest decryption of ElGamal cipher text $(c_{1}, c_{2})$ (which is $(g^r, g^m \cdot h^r)$ for some randomness $r$) 
with revealing 
our private key $x$? To solve this problem, we use a well known protocol
for proving equality of the discrete logarithm \citep{10.1007/3-540-69053-0_9}.
We first discuss the protocol, and later we will show that how we can adopt 
the protocol for our purpose. 

\textbf{Diffie Hellman Tuple:} a tuple $(g, h, u, v)$ is 
 a \textit{Diffie Hellman} tuple if there exists a $w$ such that 
 $u = g^w$ and $v = h^w$.  The protocol to prove it is:
 
 \begin{itemize}
 \item $P$ chooses a random $r$ and sends $a=g^r$ and $b = h^r$.
 \item $V$ sends a random $e$
 \item $P$ sends $z =r+ e \cdot w$
 \item $V$ check $g^z = a \cdot u^e$ and $h^z = b\cdot v^e$ 
 \end{itemize}
 

Now we come back to our original problem, i.e. proving that $m$ is the honest 
decryption of $(c_{1}, c_{2})$. From these values, we construct a \textit{Diffie Hellman} tuple
by multiplying $c_{2}$ with $g^{-m}$, i.e.
$(g, h, c_{1}, c_{2} \cdot g^{-m})$. A simple algebraic simplification shows that 
this tuple can be written as $(g, h, g^r, g^m \cdot h^r \cdot g^{-m})$ for some 
random $r$. A further simplification leads to $(g, h, g^r, h^r)$, and 
this tuple is clearly a \textit{Diffie Hellman} tuple, where $u = g^r$ and $v = h^r$. 
We could not have been able to construct a  \textit{Diffie Hellman} tuple and proved 
the equality of discrete log if we had claimed anything other than the origin value $m$. For example, 
suppose a cheating prover  claims that $m_{1}$ (different from $m$) is the honest decryption of $(c_{1}, c_{2})$.
Following the cheating prover claim, we construct the \textit{Diffie Hellman} tuple 
 $(g, h, g^r, g^{m} \cdot h^r \cdot g^{-m_{1}})$. 
 Clearly, the tuple $(g, h, g^r, h^r \cdot g^{m - m_{1}})$ is not 
\textit{Diffie Hellman} tuple; hence a cheating prover would not succeed. 




\section{Homomorphic Tally}
\label{sec:homo_tally}
Now that we have sorted out the correct decryption, the next challenge in our 
tally sheet is  computing the final tally homomorphically.  Since our encryption 
is additive ElGamal, and recall that our ballot is a matrix of ciphertexts: 

\begin{pmatrix}
  (g^{r_{11}}, g^{m_{11}} * h^{r_{11}})&  (g^{r_{12}}, g^{m_{12}} * h^{r_{12}}) & \cdots &  (g^{r_{1n}}, g^{m_{1n}} * h^{r_{1n}}) \\
 (g^{r_{21}}, g^{m_{21}} * h^{r_{21}})&  (g^{r_{22}}, g^{m_{22}} * h^{r_{22}}) & \cdots &  (g^{r_{2n}}, g^{m_{2n}} * h^{r_{2n}}) \\
  \vdots  & \vdots  & \ddots & \vdots  \\
  (g^{r_{n1}}, g^{m_{n1}} * h^{r_{n1}})&  (g^{r_{n2}}, g^{m_{n2}} * h^{r_{n2}}) & \cdots &  (g^{r_{nn}}, g^{m_{nn}} * h^{r_{nn}}) \\
 \end{pmatrix}


To compute the finally tally, all we have to do is stack all the valid
ballots (matrices) together and multiply the corresponding ciphertexts 
together to get the final tally matrix (point wise matrix multiplication). 
The final computed tally can be 
decrypted honestly by using the same principals described in the previous 
section.  We can capture all these concepts in Coq based on the 
algebraic structures, group, field, vector space, and prove all the 
properties by simple algebraic manipulation. We can represent 
encryption, decryption and ciphertext multiplication for 
a given cyclic group $(G, g, h, x)$ such that $h = g^x$:

\begin{displayquote}
$\text{elGamal\_enc (g h : G) (r : F) :=} (g^r, g^m \cdot h^r)$

$\text{elGamal\_dec (g h : G) }  (c_{1}, c_{2}) := c_{2} \cdot c_{1}^{-x}$

$\text{elGamal\_mult } (c_{1}, c_{2}) (d_{1}, d_{2}) := (c_{1} \cdot d_{1}, c_{2} \cdot d_{2})$

\end{displayquote}

\noindent
In fact, by simple algebraic manipulation, we can prove that 
decryption is left inverse of encryption. 


\begin{align}
  \text{elGamal\_dec g h (elGamal\_enc g h r)} &= \text{elGamal\_dec g h } (g^r, g^m \cdot h^r)  (unfolding) \nonumber \\
                     &= g^m \cdot h^r \cdot (g^r)^{-x}  (unfolding) \nonumber \\
                     &= g^m \cdot (g^x)^r \cdot (g^r)^{-x} (substitution) \nonumber \\
                     &=  g^m \cdot g^{xr} \cdot g^{-rx} (algebraic-simplification)\nonumber \\
                     &= g^m\nonumber 
\end{align}


\noindent 
The final decrypted tally would be a matrix filled with values like $g^{m_{1} + m_{2} + \cdots }$, and we 
need to do a search to find the values of $m_{1} + m_{2} + \cdots$ from the 
final decrypted tally. A drawback of this method is that if the number of candidates and ballots are
 large, then calculating $m_{1} + m_{2} + \cdots$ from 
	      $g^{m_{1} + m_{2} + \dotsb  + m_{n}}$ is 
	      not very practical \citep{10.1007/3-540-69053-0_9}.  


\section{IACR 2018 Election}
\label{sec:election_iacr}
We followed some of these techniques explained above write a formal 
certificate checker for  IACR 2018 directors election scrutiny sheet
\footnote{https://vote.heliosvoting.org/helios/elections/60a714ea-ce6d-11e8-8248-76b4ab96574c/view}.  
The 2018  IACR directors election considered seven 
 candidates to fill three positions on the board of directors.   The voting 
 style was approval voting where all the eligible voters, IACR members, 
 could vote for as many candidates as they liked. After the counting, 
 the top three members were elected to fill the positions. 
 
The Helios voting system \citep{Helios:2016:HVS}  was used for the election, and the system
was configured with four authorities, who generated an ElGamal \citep{elgamal1985public}  public
key such that all four authorities were required to decrypt efficiently.    
Every eligible voter received the credentials by email which they used 
to cast their ballot from their personal computer. 
During the cast process, each voter created seven ElGamal cipher texts, 
encrypting either zero or one, for the seven participating candidates. 
Since the vote was exponent, the ElGamal cryptosystem became 
homomorphic additive. At this point,  the voter is then offered the chance to audit 
her encrypted ballot to check that
it does indeed contain the vote she intended. If she chooses to audit, she must
discard this ballot and asked to  cast 
a fresh ballot. This mechanism is called called Benaloh challenge, 
and the purpose of this challenge is to catch "cheating machines". 
Moreover, this method ensures the cast-as-intended because 
a cheating machine would not know when a voter would 
cast her ballot, so, in the most likely scenario, a voter would 
end up cast her true intentions. Once she has an 
unaudited ballot with which she is happy, she casts it. 
The Helios website maintains an append-only bulletin board on which the voter's
encrypted ballot appears.  
After the voting period is over,
all the encrypted ballots corresponding to all candidates are multiplied together; so that there is 
now a single ciphertext for each candidate, encoding the number of votes for
that candidate.  The authorities then decrypt 
these (seven) ciphertexts, announce the results and prove,
using a sigma protocol, that the announced result is the 
correct decryption.


Now we focus on three aspects of verifiability: cast-as-intended, collected-as-cast, 
and tallied-as-collected. The cast-as-intended has already been assured 
by Benaloh challenges, and collect-as-cast is ensured by every voter 
checking her ballot on the bulletin board. 
The more complicated step is the counted-as-collected
check.  In order to verify the tallied-as-cast, a scrutineer has to check 
only the valid ballots (those which are encryption of either zero or one) 
has contributed to final tally, the final tally has been calculated 
correctly, and the final tally has been decrypted honestly. 

At this stage, there is a published list of encrypted ballots on the bulletin board
and a published result.  Moreover, to enable scrutiny, the election authority publishes, 
non-interactive, sigma protocol transcripts for correct encryption and decryption. 
Using these transcripts,  the scrutineer can verify the election by checking the following
three things. First, all the ballots included in final tally are indeed the encryption of 
zero or one, and any ballot containing any other value has been discarded. 
Second, the scrutineer reruns the (multiplication)
computation and checks that the resulting ciphertexts matches the published one.
Finally, she checks that the transcripts are valid for the decryption of these
combined ciphertexts with respect to the announced result.  These  three checks
suffice to ensure that the ballots were counted-as-collected. 


IACR used Schnorr group to avoid attacks various attacks on solving the discrete 
logarithm problem.  A Schnorr group is a multiplicative Abelian subgroup of prime order q of the field of 
integers modulo a prime p, where p = k * q + 1 for some integer.   In IACR election, 
the primes used were:
\begin{minted}[fontsize=\footnotesize]{coq}

Definition P : Z := 16328632084933010002384055033805457329601614771
1859553897391673090862148004064657990385836349537529416756455621824
9812075026498049238137557936767564877129380031037096474576701424363
8518442553823973482995267304044326777047662957480269391322789378384
6194285964464469846943061876447674624609656225800875643392126317758
1789595840901667639897567126617963789855768731707617721884323315069
5157881061257053019133078545928983562221396313169622475509818442661
0470184362648069010239662367183672047107559358990137503061077380023
6413791742659573740387111418775080434656473125060919684663818390398
2387884578266136503697493474682071.


Definition Q : Z := 61329566248342901292543872769978950870633559608
669337131139375508370458778917.


\end{minted}

Since theorem provers are known for proving mathematical statements, but not for being good at running 
computation inside their environment. Naturally, proving any mathematical statement, e.g. number 
theoretic proofs, which are computational intensive would not be a ideal situation for theorem provers. 
However, the recent advancement in theorem provers (specifically Coq) led
us to prove primality of two large prime numbers inside the Coq.
To begin with we utilise the CoqPrime
library\footnote{https://github.com/thery/coqprime} to prove in Coq that the
numbers used to define the Schnorr group are
in fact prime.
\begin{minted}{coq}

Lemma P_prime : prime P.

Lemma Q_prime : prime Q.

\end{minted}


Finally, we extracted the Coq code as a OCaml and wrote a main file to glue the 
extracted code and parsing code. Upon execution, the code returned 
yes, which asserts that the result produced were correct. 


\section{Summary}
\label{sec:summary_software}
In this chapter, we have sketched the ideas for developing a formally verified certificate checker
for the certificate we produced in the last chapter. However, due to time constraint and 
complexity of shuffle primitive, 
we ended up verifying a simple certification, which did not not involve any zero-knowledge-proof 
of shuffle.  Finally, in this chapter we closed the loop of decrease in number of scrutineers because 
any one can run the certificate checker. Moreover, we open sourced 
\footnote{https://github.com/mukeshtiwari/secure-e-voting-with-coq} the checker, so 
that it can be inspected by anyone (we would like to call it correctness by democratic 
process).  One thing we would like to emphasize that cryptographic concepts are 
inherently very complex, so running a certificate checker certainly not amounts 
to understanding the various bits of cryptography and formal method used to develop the
certificate checker.

In the next chapter, we will discuss some of the properties of Schulze method which we have 
formalized in Coq. 











































   
   
   
   
   
   