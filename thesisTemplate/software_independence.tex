\chapter{Scrutiny Sheet : Software Independence}
\label{cha:software_independence}

\epigraph{Somewhere inside of all of us is the power to change the world.} 
{\textit{Roald Dahl }}

\section{Introduction}


One of the major disadvantage, as we discusses in the last chapter, of introducing cryptography 
to achieve privacy and verifiability, encryption to make the 
content of ballot private and zero-knowledge-proof for verification of claims, makes the verification 
process cumbersome. As a consequence, the verification process (checking the scrutiny sheet) is only accessible 
to tiny fraction of representative population, mainly cryptographers, results into a sharp decrease in number of scrutineers. 
While it is not very difficult to find cryptographers in any economy, 
they are not in abundance, and in addition, they are not representative of democracy. 
In order to increase the number of scrutineers, we follow the route of providing a formally verified open-source 
reference certificate checker, which anyone can inspect and run on the election data. 
  The rationale behind formally verified certificate checker is \emph{correctness}, 
  and open-source is to gain the public trust  via scrutiny or openness.  
  For example, consider a scenario where we do not provide the reference checker,
  then how 
  likely would it be for community/voters to develop the 
  verified checker? Moreover, assuming that we publish one unverified certificate hecker,
  what would happen if it returns false on a valid certificate because of its own bug? 
  Both situations, off course, would be a devastating situation, so not only we 
  should provide a reference certificate checker, but it should also be a formally verified one. 
  Additionally, a formally verified reference certificate checker would open the gate for
  debate in case of someone's implementation for checking certificate diverges from the reference checker.  
In the case of diverging situation, there are two possibility, either the reference checker is verified 
using wrong assumption,  or the implementation itself is wrong.  The first situation is certainly 
not very pleasant because it would deteriorate the public trust in the system, but nonetheless, it is always
good to  have openness in democracy to make it more strong. 
  
  
  
  At this point of time, the astute reader can criticize us that the source code of formally verified certificate checker should be 
  inspected by someone having the expertise of cryptography and formal verification (logic) 
  to see if the verification has been carried out diligently. We sincerely accept the criticism of the 
  reader that this is indeed the case, but what makes this effort  worthwhile is 
  the increased ability of voters to verify the election by themselves by 
  simply running the checker, which has been checked by a community having the 
  expertise of cryptography and logic. 

\textbf{Chapter Outline:} 


In this chapter, we flesh down all the concepts required to develop a certificate  in the certificate we generated 


\begin{enumerate}
\item Diffie-Hellman Tuple for honest decryption
\item Homomorphic encryption
\item Sigma Protocol
\item Pedersan Commitment 
\item Wikstrom shuffle 
\end{enumerate}


\begin{enumerate}
\item Need for Scrutiny sheet (Electronic voting)
\item General Structure of Scrutiny sheet
\item How to verify it
\item Extracting a OCaml code
\item Why should we trust the extracted code
\item Bits and pieces need to write scrutiny checker
\item Coq formalization
\item Explain little bit about Coq
\item Why do we need scrutiny checker
\item 
\end{enumerate}



We give a brief overview of the needed concepts to understand our scrutiny sheet.


























        

  
\section{Summary}
   Write some advantage of proof checker for certificates.
   To create the mass scrutineers, all we need is a simple proof checker
   which would take proof certificate as input and spit true or false.
   If it's true then we accept the outcome of election otherwise something 
   wrong.
 










































   
   
   
   
   
   