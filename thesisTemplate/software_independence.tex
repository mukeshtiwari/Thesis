\chapter{Scrutiny Sheet : Software Independence (or Universal Verifiability)}
\label{cha:software_independence}

\epigraph{Somewhere inside of all of us is the power to change the world.} 
{\textit{Roald Dahl }}

Getting everything right in electronic voting is very difficult, and assuming, for a moment,
that everything is correct, convincing this fact to every stakeholder and  any independent third party 
is almost next to impossible.  As we stated in earlier chapter [Give a link to the section] that 
software is complex artefact and, often, poorly design and tested. It should not come as a surprise to anyone
that we are far more competent in producing the incorrect software than producing a provable correct one. 
 In order to tackle the software complexity problem and ensure the 
public trust in process, Ronald Rivest and John Wack  proposed  the term "software-independence": 
\footnote{https://people.csail.mit.edu/rivest/RivestWack-OnTheNotionOfSoftwareIndependenceInVotingSystems.pdf}

\textit{A voting system is software independent if an undetected change or error
   in its software can not cause an undetected change or error in an 
   election outcome.}
 
 
 Software-independence is weaker notion than  end-to-end verifiability. Software independence 
 put forward the idea of detection (and possible correction) of outcome of election due to 
 software bug, while the end-to-end verifiability makes the whole process transparent without trusting 
 any component involved in the process (including any hardware and software) \citep{Bernhard:2017:PES}.
 Recall that end-to-end verifiability:
  \begin{itemize}
  \item Cast-as-intended: Every voter can verify that their ballot was cast as
  intended
  \item Collected-as-cast: Every voter can verify that their ballot was collected as
  cast
  \item Tallied-as-cast: Everyone can verify final result on the basis of the
  collected ballots.
\end{itemize}


\cite{Benaloh:2006:SVE:1251003.1251008} has given a detailed overview  about achieving each step of 
end-to-end verifiability, we are only concern about the last phase, i.e. \texttt{Tallied-as-cast}.
Scrutiny sheet not only provides the \texttt{Tallied-as-cast}
(also known as universal verifiability) assuming that first two, \texttt{Cast-as-intended} and 
\texttt{Collected-as-cast}, hold, but it makes our voting system  software independent. 
It is worth noting that any bug or malicious behaviour in 
software used to produce the result can not go undetected if the results produced were incorrect. The 
rationale is that any independent third party auditing/verifying the result would
write his own checker to accomplish the task, and statistically, if more people are auditing the election, then 
it is highly unlikely that incorrect result produced by buggy/malicious software would survive for long. 



\textbf{Chapter Overview}[Possibly rewriting after finishing the chapter]
In the closing remark of last chapter, we argued that it is always a good idea to provide a open source 
reference checker. This chapter focuses on the technical details needed to develop a certified checker 
for election conducted on encrypted ballots. 
In section [refer the section], we discuss the structure of  encrypted-ballot scrutiny sheet, 
elaborate the relevant details to understand the certificate, section [some number] discusses 
about what it means to verify the zero-knowlege-proof of different statement. Developing a 
formally verified certificate checker for our certificate could have taken long time, so in order 
to demonstrate the idea we have the taken the IACR 2018 election which we discuss in section [some number].
%
%We then discuss the technical details needed to develop checker for plaintext-ballot scrutiny sheet, and 
%encrypted-ballot scrutiny sheet. In the end, we briefly discuss a formally verified checker for 
%IACR 2018 election. 

\begin{enumerate}
\item first paste the certificate
\item take each piece of information, and show that how can it be verified
\item Flesh the details of honest decryption zero knowledge proof
\item Flesh the details of permutation shuffle
\end{enumerate}

%Same as the last chapter, introduce the motivation and the high-level picture to
%readers, and introduce the sections in this chapter. 
%
%One of the crucial 
%aspect in electronic voting is ability to detect change in outcome of 
%election because of software bugs. 


%   A voting system is software independent if an undetected change or error
%   in its software can not cause an undetected change or error in an 
%   election outcome.
%\section{Verification and Verifiability}
%  Paste Evote section 2

\section{Certificate : Ingredient for Verification}
  \fix{There are two notion of verification. Software verification and 
   election verification. A electronic voting scheme implemented in 
   Coq is verified implementation, but it does not imply that method 
   itself is verifiable. Explain the software independence } 
   
   \subsection{Plaintext Ballot Certificate}
    Flesh out the details needed here for writing proof checker
    
    \subsection{Encrypted Ballot Certificate}
    Flesh out the details needed here for writing proof checker
    
   
\section{Proof Checker}
  Write the details of proof checker for both certificates
  
  	\subsection{A Verified Proof Checker : IACR 2018}
  	Write some details about proof checker.
  
\section{Summary}
   Write some advantage of proof checker for certificates.
   To create the mass scrutineers, all we need is a simple proof checker
   which would take proof certificate as input and spit true or false.
   If it's true then we accept the outcome of election otherwise something 
   wrong.