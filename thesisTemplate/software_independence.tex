\chapter{Scrutiny Sheet : Software Independence}
\label{cha:software_independence}

\epigraph{Somewhere inside of all of us is the power to change the world.} 
{\textit{Roald Dahl }}

\section{Introduction}


One of the major disadvantage of introducing cryptography 
to achieve privacy and verifiability, encryption to make the 
content of ballot private and zero-knowledge-proof for verification of claims, makes the verification 
process cumbersome. As a consequence, the verification process (checking the scrutiny sheet) is only accessible 
to tiny fraction of representative population, cryptographers, results into a sharp decrease in number of scrutineers. 
While it is not very difficult to find cryptographers to verify the election, 
they are, off course, not the representative population of any democracy. 
In order to increase the number of scrutineers and subsequently confidence in electronic voting, we follow the 
route of providing a formally verified open-source 
reference certificate checker, which anyone can inspect and run on the election data. 
  The rationale behind formally verifying the certificate checker is \emph{correctness}
  and open sourcing is to gain the public trust  via scrutiny or openness.  
  For example, consider a scenario where we do not provide the reference checker,
  then how 
  likely would it be for community/voters to develop the 
  verified checker? Moreover, assuming that we publish one unverified certificate checker,
  what would happen if it returns false on a valid certificate because of its own bug? 
  Both situations, off course, would be a devastating situation, so not only we 
  should provide a reference certificate checker, but it should be a formally verified one. 
  Additionally, a formally verified reference certificate checker would open the gate for
  debate in case of someone's implementation for checking certificate diverges from the reference checker.
In the case of diverging situation, there are two possibility, either the reference checker is verified 
using wrong assumptions,  or the implementation itself is wrong.  The first situation is certainly 
not very pleasant because it would deteriorate the public trust in the system, but nonetheless, it is always
good to  have openness in democracy to make it more strong. 
  
  
  
  At this point of time, the astute reader can criticize us that the source code of formally verified certificate checker should be 
  inspected by someone having the expertise of cryptography and formal verification (logic) 
  to see if the verification has been carried out diligently. We sincerely accept the criticism of the 
  reader that this is indeed the case, but what makes this effort  worthwhile is 
  the increased ability of voters to verify the election by themselves by 
  simply running the checker, which has been checked by a community having the 
  expertise of cryptography and logic. 


In this chapter, we will discuss all the concepts required to develop a verified certificate checker for the certificate we generated 
in the last chapter.  We have already explained our certificate in the section \ref{sec:extract}, but intuitively, 
checking our certificate amounts to proving that  the homomorphic margin has been computed correctly and zero-knowledge-proof for every claim is correct. 
Recall that our claims were that we have honestly decrypted every value, computed the finally tally correctly, and  
a ballot has been permuted by the same permutation whose commitment is published (commitment consistent shuffle \citep{Wikstrom:2009:CPS}).
We will sketch the proof of honest decryption, Pedersen's commitment, and homomorphic computing of 
tally. As a consequence, translating these sketches into Coq proof is fairly trivial. The triviality comes from 
the fact that most of these proofs are basic algebraic manipulation over various algebraic structures, 
group, and field. Furthermore, we have formalized these algebraic manipulation
in Coq. We have not formalized the shuffle algorithm \citep{Wikstrom:2009:CPS} because of 
its complexity and would take a fair amount of time (the efforts are ongoing).  Finally, all 
of the concepts explained in this chapter would be sufficient to develop the formally verified 
certificate checker for IACR election. 

\textbf{Chapter Outline:} In the section $\ref{sec:algebra}$, we discuss the underlying algebraic structures needed for various 
 cryptographic operations. Section $\ref{sec:pedersen}$ focuses on generalizing the Pedersen commitment scheme for 
 a matrix. In the following section $\ref{sec:sigma_coq}$, we discuss the details of sigma protocol, and its formalization 
 in Coq. Moreover, we also discuss how to turn a probabilistic reasoning, which is the heart of sigma protocols, to 
 deterministic reasoning without losing the meaning of sigma protocol. In section $\ref{sec:conc_sigma}$, we show 
 how to make a concrete instance of sigma protocol by giving an example of discrete logarithm. In addition, 
 we show the protocol needed for honest decryption (section $\ref{sec:dec_sigma}$). Section $\ref{sec:homo_tally}$ sketches 
 the homomorphically tally based on additive ElGamal scheme. Finally, we discuss the IACR scrutiny sheet in section $\ref{sec:election_iacr}$, 
 and we summarize the chapter in the section $\ref{sec:summary}$


\section{Algebraic Structures: Building Blocks}
\label{sec:algebra}
The basic building blocks any cryptographic system are algebraic structures, specifically cyclic group of prime order, field and vector space. In general, 
we do not need vector space, and group and field are sufficient for most of the cryptographic purposes. However,
vector space  of a cyclic group of prime order over the field of integers modulo the same order is nicer to work 
because multiplying a group element by a field element can be abstracted over scalar multiplication of vector space.

\begin{definition}[Group] 
A group is a set $G$, with a binary operator $* : G \rightarrow G \rightarrow G$, identity element $e$, and inverse operator $inv : G \rightarrow G$ such 
    that the following laws hold:  \end{definition} 
    \begin{itemize}
     \item \texit{Associativity}: $\forall$  a b c $\in G,$  $a * (b * c) = (a * b) * c$
    \item \texit{Closure}: $\forall$ a b $\in G,$  $a * b \in G$
    \item \texit{Inverse Element}: $\forall$ a $\in G$ $\exists$ $a^{-1} \in G$, such that $a * a^{-1} = a^{-1} * a = e$. $a^{-1}$ is called inverse of $a$ ($inv$ a).
    \item \texit{Identity}: $\forall$ a $\in G,$  $a * e = e * a  = a$
    \end{itemize}
  
    \noindent
    If the group is commutative, i.e. $\forall$ a b $\in  G,$  $a * b = b * a$, then we call it abelian group.  We can represent the abelian group in Coq by using the 
    record data type. 
 
 \begin{verbatim}

Record AbelGroup (G : Type) 
    (Hdec : forall x y : G, {x = y} + {x <> y}) 
    (dot : G -> G -> G)  (inv : G -> G) (e : G) :=
  {
    dot_associativity : forall x y z, 
      dot x (dot y z) = dot (dot x y) z;
    dot_left : forall x, dot x e = x;
    dot_right : forall x, dot e x = x;
    left_inverse : forall x, dot (inv x) x = e;
    right_inverse : forall x, dot x (inv x) = e;
    commutative : forall x y, dot x y = dot y x
  }.
  
\end{verbatim}    

\noindent
Our Coq encoding is slight different from the definition of the group we gave above, mainly that \textit{G : Type} "read it as A has the type Type" and 
\textit{(Hdec : forall x y : G, {x = y} + {x <> y})}. Because of the theoretical foundations of Coq, every term in Coq has to have a type;
hence we need to explicitly state the type of term $G$, and we assume a decidable equality $Hdec$ on the elements of $G$.  

\begin{definition}[Field] 
A field  is a set $F$, with two binary operator $+ : F \rightarrow F \rightarrow F$,  and $\cdot : F \rightarrow F \rightarrow F$, 
two identity element $0$ and $1$, and two unary operator $- : F \rightarrow F$, $1/ : F \rightarrow F$  such that:
\end{definition} 
 \begin{itemize}
 \item (F, +, 0, -) forms an abelian group.
 \item (F - $\lbrace 0 \rbrace$, $\cdot$, 1, 1/) forms an abelian group.
 \item $\cdot$ distributes over +.
 \end{itemize}
 
 

\begin{definition}[Vector Space]
A set $V$ with two binary operation $+ : V \rightarrow V \rightarrow V$ and $\cdot : \mathbb{F}  \rightarrow V \rightarrow V$, 
is  a vector space over a field $\mathbb{F}$ if the following properties hold:
\end{definition} 
\begin{itemize}
 \item Closure under addition: (V, +) forms an abelian group. 
 \item Scalar  distribute  with respect  to  vector  addition: $\forall$  $r \in \mathbb{F}$, $u$ $v$ $\in V$,  $r \cdot (u + v) = r \cdot u + r \cdot v$
 \item Scalar distribute with respect to field addition: 
                $\forall$  $a$ $b$ $\in \mathbb{F}$, $u$ $\in V$, $(a +_{\mathbb{F}} b) \cdot u = a \cdot u + b \cdot v$
  \item Scalar multiplication is associative with respect to field multiplication:
         $\forall$ $a$ $b$ $\in  \mathbb{F}$, $u$ $\in V$ , $(a \cdot_{\mathbb{F}} b) \cdot u = a \cdot (b \cdot u)$
  \item Identity: $\forall a \in V, 1_{\mathbb{F}} \cdot a = a$

\end{itemize}



\section{Pedersen Commitment Scheme}
\label{sec:pedersen}
Recall that in the last chapter to prove that if a ballot was valid or invalid, we generated a secret permutation $\pi$ and published its commitment using 
Pedersen commitment scheme. Later, we use this to shuffle each row and column of the ballot by this (secret) permutation. 
In the shuffle algorithm \citep{Wikstrom:2009:CPS}), the data structure of permutation $\pi$ is matrix (a permutation matrix to be precise).
In this section, we discuss that how to use generalize Pedersen commitment scheme for matrix. 

A Pedersen commitment for any two give group element $g$ $h$ $\in G$, a message $m \in \mathbb{F}$, and random element 
$r \in \mathbb{F}$ is:

\begin{displayquote}

$\text{Pedersen\_commitment  g h m r := } g^r \cdot h^m$  

\end{displayquote}

We can extend the Pedersen commitment for a group element $g$, 
vector of $n$ group element $h_{1}, h_{2} \dots h_{n}$, vector of $n$ 
field element $m_{1}, m_{2} \dots m_{n}$, and a random field element 
$r$ is:

\begin{displayquote}

$\text{Pedersen\_generalized g hs ms r := } g^r \cdot \prod_{i = 1}^n  h_{i}^{m_{i}}$  

\end{displayquote}

The $\prod_{i = 1}^n  h_{i}^{m_{i}}$ function can be written in Coq as 
(this program is written in tactic mode):

\begin{verbatim}
Theorem pedersen_generalized : forall (n : nat), 
   Vector.t G (S n) -> Vector.t F (S n) -> G.
Proof.
  induction n.
  intros h m .
  inversion h;
    inversion m.
  exact (h0 ^ h1).
  intros hs ms.
  inversion hs; 
    inversion ms.
  exact ((h0 ^ h) * 
  IHn H0 H2).
Defined.
\end{verbatim}

Finally, to commit a matrix of size $NxN$, we need to call 
$Pedersen\_generalized$ on each column of matrix which 
would produce a vector of commitment of size N.


 



\section{Sigma Protocol: Efficient Zero-Knowledge-Proof}
\label{sec:sigma_coq}
A sigma protocol is a two party protocol, a prover $P$ and a verified $V$, where prover $P$ try to convince the $V$ that he 
holds a private input $x$ for some public input $w$ such that a binary relation $R$ holds, i.e. $(x, w) \in R$.  Sigma protocol, 
in general, is a three step protocol:
\begin{enumerate}
\item Initialisation: $P$ generates a random message $r$, commits it, and send the committed message to $V$
\item Challenge: $V$ generates a random message $c$, and sends it to $V$
\item Response: $P$ sends a response $z$ to $V$
\end{enumerate} 

\noindent
Upon receiving the response $z$, $V$ either accepts the proof or rejects the proof.  Now we define the sigma protocol in 
Coq by using record data type.

\begin{verbatim}
Record SigmaProtocol (Statement : Type) (* Statement x *)
       (Witness : Type) (* witness w *)
       (Rel : Statement -> Witness -> bool) (* decidable relation *)
       (RandCoin : Type) (* random coin *) 
       (Commitment : Type) (* commitments *)
       (Challenge : Type) (* challenges *) 
       (Response : Type) (* response *) :=
  MkSigma 
    {
      (* initial commitment send by the Prover *)
      initial : RandCoin -> Commitment;
      (* Randomness send by the verifier.  *) 
      challenge : Challenge;
      (* response generate by prover *)
      response : Statement -> Witness ->
                 RandCoin -> Challenge ->
                 Response;
      (* verify the response *)
      verify : Statement * Commitment * Challenge * Response -> bool;
      (* Simulator *)
      simulator : Statement -> Challenge -> Response ->
                  Statement * Commitment * Challenge * Response;
      (* Extractor *)
      extractor : Challenge -> Response -> Challenge -> Response -> Witness;
  

      (* Completness *)
      Completness : forall (s : Statement) (w : Witness) (r : RandCoin)
                      (e : Challenge),
          Rel s w = true -> verify (s, initial r, e, response s w r e) = true;

      (* Special Soundness *)
      Special_Soundness : forall s c e1 e2 r1 r2,
          e1 <> e2 ->
          verify (s, c, e1, r1) = true ->
          verify (s, c, e2, r2) = true ->
          Rel s (extractor e1 r1 e2 r2) = true;
  }.
\end{verbatim}


\noindent
The record \textit{SigmaProtocol} is indexed by \textit{Statement}, the public input 
known to $P$ and $V$, \textit{Witness}, secret input known to the $P$, 
\textit{Rel} such that $(x, w) \in Rel$, \textit{RandCoin}, the private 
random coin toss of $P$, \textit{Commitment}, commitment computed 
by $P$ based on the random coin toss of itself, 
\textit{Challenge}, the random challenge of $V$ to $P$, 
\textit{Response}, the response of $P$ send to $V$. Moreover, 
the body of record \textit{SigmaProtocol} contains
functions \textit{initial}, \textit{challenge}, \textit{response}, and \textit{verify}
to reflect the three steps of sigma protocol with two auxiliary 
functions \textit{simulator} and \textit{extractor}.  The 
reason for having \textit{simulator} and \textit{extractor} 
to prove the property of special honest verifier zero knowledge proof 
(still not mentioned in the record) and special soundness of 
sigma protocol. Finally, we have two property \textit{Completeness}, 
which states that if $P$ and $V$ follow the protocol, then 
verifier would accept the proof and \textit{Special\_Soundness},
which states that if $P$ is able to convince $V$ with two 
accepting transcript for the same commitment, then $V$ can extract the witness. 


\noindent
Our definition of sigma protocol is almost complete except we have not given 
the special honest verifier zero knowledge proof axiom.  Recall that special honest 
verifier zero knowledge proof amounts to a probabilistic polynomial time simulator 
$S$ which would generate a proof transcript for some statement 
$s$  with same probability distribution as if there were a real 
conversation between a prover $P$ and a verifier $V$ 
for the statement $s$ and witness $w$ such that $(s, w) \in  R$.
Informally, the real proof transcript depends on statement $s$, witness $w$,  
and challenge $e$, 
while the simulated proof transcript depends on statement $s$ and challenge $e$. 
(simulator does have not access to witness $w$, so to generate a accepting 
proof just by using $s$ and $e$, it uses a concept call rewinding.)


If we were doing probabilistic reasoning, we could have represent the real view and 
simulator's view as following (not a real code, but inspired by Certicrypt):

\begin{verbatim}
Real_view (s : Statement) (w : Witness) (e : Challenge) := do
  r <- G (* generate a random value from some group G *)
  let a := g^r (* commitment *)
  let z := r + e * w  (* compute the response *)
  return (s, a, e, z)
\end{verbatim}


\begin{verbatim}
Simulator_view (s : Statement) (e : Challenge) := do
 z <- G (* random element from some group G *)
 return (s, g^r * s^(-e), e, z)
\end{verbatim}

\begin{verbatim}

 Special_Honest_Verifier_ZKP  (s : Statement) (w : Witness) 
  (e : Challenge) (H : Rel s w = true) := 
  Pr [Real_view  s w e] = 
  Pr [Simulator_view s e]

\end{verbatim}

\noindent
Based on these two probabilistic view, we could have shown that two are equal. Intuitively, 
in $Real\_view$, statement $s$, witness $w$, and the challenge $e$ is fixed, so the only thing that vary is the
random coin drawn uniformly from some group $G$. The probability of drawing an element uniformly 
from a given set $S$ is 1/|S| (| | represent the cardinality of a set), so the value of Pr [Real\_view  s w e] = 1/|G|. 
By the same token of reasoning, Pr [Simulator\_view s e] = 1/|G| which concludes that 
both probabilities are equal. However, the only problem is that we are not doing probabilistic reasoning, 
but not all all hope is lost.
One key observation to escape this probabilistic reasoning is 
that we can make the randomness $r$ and response $z$ as 
explicit parameter. Consequently, our probabilistic program would turn into 
a deterministic Coq program. 


\begin{verbatim}
Real_view (s : Statement) (w : Witness) (e : Challenge) 
  (r : RandCoin):=
  let a := g^{r} in (* commitment *)
  let z := r + e * w in  (* compute the response *)
  (s, a, e, z)
\end{verbatim}


\begin{verbatim}
Simulator_view (s : Statement) (e : Challenge) 
 (z : Response) :=
 (s, g^r s^(-e), e, z)

\end{verbatim}



Now that our views are deterministic, we need to find a way to model 
the probability distribution of the two views. We solve this problem 
by showing a bi-implication  between  the real view and simulated view.
We show that real view and simulator's view align with each other. 
In terms of Coq, it is expressed as:

\begin{verbatim}
 (* Probablistic Reasoning could have made it nicer *)
 Special_Honest_Verifier_ZKP  (s : Statement) (w : Witness) (e : Challenge):
  forall (r : RandCoin), verify (Real_view s w e r) = true <->
  forall (z : Response), verify (Simulator_view s e z) = true;

\end{verbatim}

The axiom $Special\_Honest\_Verifier\_ZKP$ says that for any  given 
fixed statement $s$, witness $w$, challenge $e$ and assumption that $Rel$ $s$ $w$ holds, 
then for every random coin $r$  and a accepting real transcript, simulator can construct 
an accepting transcript from all random response drawn from response space. 
We add this axiom to the record data type $SigmaProtocol$ to make it 
complete. 

Finally, we can compose different  
sigma protocol using our definition. For example, 
we can define parallel composition, AND composition,
EQ composition, OR composition, etc.

\subsection{Concrete Sigma Protocol: Discrete Logarithm}
\label{sec:conc_sigma}
One of the most basic sigma protocol is proof of knowledge of 
discrete logarithm, i.e. given two elements$g$ and $h$ of 
a group $G$, prover convinces the verifier that 
he knows the discrete logarithm ($\log_g h$) in zero 
knowledge. In mathematical 
notation of zero knowledge proof, it is represented as:
$ZKPoK \lbrace w \text{ | } h = g^w \rbrace$. We can show 
that this is a sigma protocol inside Coq by 
by encoding all the  functions 
and proving all the axioms mentioned in 
our record type $SigmaProtocol$.  For example,
 we can write the $initial$ function as taking a input 
random coin $r$ as input and computing 
$g^r$, $challenge$ as a function which simply returns 
a challenge $e$, and so forth: 


\begin{displayquote}

$\text{initial r := } g^r}$  

$\text{challenge := } e$

$\text{response h w r e := } r + e \cdot w$

$\text{verify h a e z  := } g^z = a \cdot h^e$

$\text{simulator s e z := } (g^z \cdot h^{-e}, e, z)$

$\text{extractor }  $c_{1}$ $z_{1}$ $c_{2}$ $z_{2}$ := (z_{1} - z_{2}) \cdot (c_{2} - c_{1})^{-1}$

\end{displayquote}

Based on these definitions, we can easily discharge the three proofs, \textit{Completeness}, 
\textit{Special Soundness}, and \textit{Special Honest Verifier Zero Knowledge} axiom by simple 
algebraic manipulation.

\subsection{Honest Decryption: Diffie Hellman Tuple}
\label{sec:dec_sigma}
We have sigma protocol at our arsenal, we focus on honest decryption 
problem. How can we convince someone that for a given group 
$(G, g, p, h)$ and private key $x$ ($h := g^x$), the message $m$ is the 
honest decryption of ElGamal cipher text $(c_{1}, c_{2})$ with revealing 
our private key $x$? To solve this problem, we use a well known protocol
for proving equality of the discrete logarithm \citep{10.1007/3-540-69053-0_9}.
We first discuss the protocol, and later we will show that how we can adopt 
the protocol for our purpose. 

\textbf{Diffie Hellman Tuple:} a tuple $(g, h, u, v)$ is 
 a \textit{Diffie Hellman} tuple if there exists a $w$ such that 
 $u = g^w$ and $v = g^v$.  The protocol to prove it is:
 
 \begin{itemize}
 \item $P$ chooses a random $r$ and sends $a=g^r$ and $b = h^r$.
 \item $V$ sends a random $e$
 \item $P$ sends $z =r+ e \cdot w$
 \item $V$ check $g^z = a \cdot u^e$ and $h^z = b\cdot v^e$ 
 \end{itemize}
 

Now we come back to our original problem, i.e. proving that $m$ is the honest 
decryption of $(c_{1}, c_{2})$. From these values, we construct a \textit{Diffie Hellman} tuple
by multiplying $c_{2}$ with $g^{-m}$, i.e.
$(g, h, c_{1}, c_{2} \cdot g^{-m})$. A simple algebraic simplification shows that 
this tuple can be written as $(g, h, g^r, g^m \cdot h^r \cdot g^{-m})$ for some 
random $r$. A further simplification leads to $(g, h, g^r, h^r)$, and 
this tuple is clearly a \textit{Diffie Hellman} tuple, where $u = g^r$ and $v = h^r$. 
We could not have been able to construct a  \textit{Diffie Hellman} tuple and proved 
the equality of discrete log if some would have claimed otherwise. For example, 
suppose a cheating prover  claims that $m$ is the honest decryption of $(c_{1}, c_{2})$,
while it is not. 
If $m$ is not the honest decryption of $(c_{1}, c_{2})$, then we would end up with 
a tuple $(g, h, g^r, g^{m_{1}} \cdot h^r \cdot g^{-m})$, where $m_{1}$ is the honest 
decryption of $(c_{1}, c_{2})$.  Clearly, the tuple $(g, h, g^r, h^r \cdot g^{m_{1} - m})$ is not 
\textit{Diffie Hellman} tuple; hence a cheating prove would not succeed. 




\section{Homomorphic Tally}
\label{sec:homo_tally}
Now that we have sorted out the correct decryption, the next challenge in our 
tally sheet is  computing the final tally homomorphically.  Since our encryption 
is additive ElGamal, and recall that our ballot is matrix of ciphertext: 

\begin{pmatrix}
  (g^{r_{11}}, g^{m_{11}} * h^{r_{11}})&  (g^{r_{12}}, g^{m_{12}} * h^{r_{12}}) & \cdots &  (g^{r_{1n}}, g^{m_{1n}} * h^{r_{1n}}) \\
 (g^{r_{21}}, g^{m_{21}} * h^{r_{21}})&  (g^{r_{22}}, g^{m_{22}} * h^{r_{22}}) & \cdots &  (g^{r_{2n}}, g^{m_{2n}} * h^{r_{2n}}) \\
  \vdots  & \vdots  & \ddots & \vdots  \\
  (g^{r_{n1}}, g^{m_{n1}} * h^{r_{n1}})&  (g^{r_{n2}}, g^{m_{n2}} * h^{r_{n2}}) & \cdots &  (g^{r_{nn}}, g^{m_{nn}} * h^{r_{nn}}) \\
 \end{pmatrix}


To compute the finally tally, all we have to do is stack all the "valid" 
ballots (matrices) together and multiply the corresponding ciphertexts 
together to get the final tally matrix (point wise matrix multiplication). 
The final computed tally can be 
decrypted honestly by using the same principals described in the previous 
section.  We can capture all these concepts in Coq based on the 
algebraic structures, group, field, vector space, and prove all the 
properties by simple algebraic manipulation. We can represent 
encryption, decryption and ciphertext multiplication for 
a given cyclic group $(G, g, h, x)$ such that $h = g^x$:

\begin{displayquote}
$\text{elGamal\_enc (g h : G) (r : F) :=} (g^r, g^m \cdot h^r)$

$\text{elGamal\_dec (g h : G) }  (c_{1}, c_{2}) := c_{2} \cdot c_{1}^{-x}$

$\text{elGamal\_mult } (c_{1}, c_{2}) (d_{1}, d_{2}) := (c_{1} \cdot d_{1}, c_{2} \cdot d_{2})$

\end{displayquote}

\noindent
In fact, by simple algebraic manipulation, we can prove that 
decryption is left inverse of encryption. 


\begin{align}
  \text{elGamal\_dec g h (elGamal\_enc g h r)} &= \text{elGamal\_dec g h } (g^r, g^m \cdot h^r)  (unfolding) \nonumber \\
                     &= g^m \cdot h^r \cdot (g^r)^{-x}  (unfolding) \nonumber \\
                     &= g^m \cdot (g^x)^r \cdot (g^r)^{-x} (substitution) \nonumber \\
                     &=  g^m \cdot g^{xr} \cdot g^{-rx} (algebraic-simplification)\nonumber \\
                     &= g^m\nonumber 
\end{align}


\noindent 
The final decrypted tally would be a matrix filled with values like $g^{m_{1} + m_{2} + \cdots }$, and we 
need to do a search to find the values of $m_{1} + m_{2} + \cdots$ from the 
final decrypted tally. As a consequence, if the number of candidates and ballots are large then 
this method is no longer a practical method. 

\section{IACR 2018 Election}
\label{sec:election_iacr}
We followed some of these techniques explained above write a formal 
certificate checker for  IACR 2018 directors election scrutiny sheet.  
The 2018  IACR directors election considered seven 
 candidates to fill three positions on the board of directors.   The voting 
 style was approval voting where all the eligible voters, IACR members, 
 could vote for as many candidates as they liked. After the counting, 
 the top three members were elected to fill the positions. 
 
The Helios voting system \citep{Helios:2016:HVS}  was used for the election, and the system
was configured with four authorities, who generated an ElGamal \citep{elgamal1985public}  public
key such that all four authorities were required to decrypt efficiently.    
Every eligible voter received the credentials by email which they used 
to cast their ballot from their person computer. 
During the cast process, each voter created seven ElGamal cipher text, 
encrypting either zero or one, for the seven participating candidates. 
Since the vote was exponent, the ElGamal cryptosystem became 
homomorphic additive. At this point,  the voter is then offered the chance to audit 
her encrypted ballot to check that
it does indeed contain the vote she intended. If she chooses to audit, she must
discard this ballot and asked to  cast 
a fresh ballot. This mechanism is called called Benaloh challenge, 
and the purpose of this challenge is to catch a cheating machines. 
Moreover, this method ensures the cast-as-intended because 
a cheating machine would have no idea when a voter would 
cast her ballot, so, in most likely scenario, a voter would 
end up cast her true intentions. Once she has an 
unaudited ballot with which she is happy, she casts it. 
The Helios website maintains an append-only bulletin board on which the voter's
encrypted ballot appears.  
After the voting period is over,
all the encrypted ballots corresponding to each candidate are multiplied together; so that there is 
now a single ciphertext for each candidate, encoding the number of votes for
that candidate.  The authorities then decrypt 
these (seven) ciphertexts, announce the results and prove,
using a sigma protocol, that the announced result is the 
correct decryption.


Now we focus on three aspects of verifiability: cast-as-intended, collected-as-cast, 
and tallied-as-collected. Well, the cast-as-intended has already been assured 
by Benaloh challenges, and collect-as-cast is ensured by every voter 
checking her ballot on the bulletin board. 
The more complicated step is the counted-as-collected
check.  In order to verify the tallied-as-cast, a scrutineer has to check 
only the valid ballots (those which are encryption of either zero or one) 
has contributed to final tally, the final tally has been calculated 
correctly, and the final tally has been decrypted honestly. 

At this stage, there is a published list of encrypted ballots on the bulletin board
and a published result.  Moreover, to enable scrutiny, the election authority publishes, 
non-interactive, sigma protocol transcripts for correct encryption and decryption. 
Using these transcripts,  the scrutineer can verify the election by checking the following
three things. First, all the ballots included in final tally are indeed the encryption of 
zero or one, and any ballot containing any other value has been discarded. 
Second, the scrutineer reruns the (multiplication)
computation and checks that the resulting ciphertexts matches the published one.
Finally, he checks that the transcripts are valid for the decryption of these
combined ciphertexts with respect to the announced result.  These  three checks
suffice to ensure that the ballots were counted-as-collected. 


IACR used Schnorr group to avoid attacks various attacks on solving the discrete 
logarithm problem.  A Schnorr group is a multiplicative Abelian subgroup of prime order q of the field of 
integers modulo a prime p, where p = k * q + 1 for some integer.   In IACR election, 
the primes used were:
\begin{lstlisting}[frame=single,basicstyle=\ttfamily\scriptsize]

Definition P : Z := 16328632084933010002384055033805457329601614771
1859553897391673090862148004064657990385836349537529416756455621824
9812075026498049238137557936767564877129380031037096474576701424363
8518442553823973482995267304044326777047662957480269391322789378384
6194285964464469846943061876447674624609656225800875643392126317758
1789595840901667639897567126617963789855768731707617721884323315069
5157881061257053019133078545928983562221396313169622475509818442661
0470184362648069010239662367183672047107559358990137503061077380023
6413791742659573740387111418775080434656473125060919684663818390398
2387884578266136503697493474682071.


Definition Q : Z := 61329566248342901292543872769978950870633559608
669337131139375508370458778917.


\end{lstlisting}

Since theorem provers are known for proving mathematical statements, but not for being good at running 
computation inside their environment. Naturally, proving any mathematical statement, e.g. number 
theoretic proofs, which are computational intensive would not be a ideal situation for theorem provers. 
However, the recent advancement in theorem provers (specifically Coq) led
us to prove primality of two large prime numbers inside the Coq.
To begin with we utilise the CoqPrime
library\footnote{https://github.com/thery/coqprime} to prove in Coq that the
numbers used to define the Schnorr group are
in fact prime.
\begin{lstlisting}[frame=single,basicstyle=\ttfamily\scriptsize]

Lemma P_prime : prime P.

Lemma Q_prime : prime Q.

\end{lstlisting}


Finally, we extracted the Coq code as a OCaml and wrote a main file to glue the 
extracted code and parsing code. Upon execution, the code returned 
yes, which asserts that the result produced were correct. 


\section{Summary}
\label{sec:summary}
In this chapter, we have sketched the ideas for developing a formally verified certificate checker
for the certificate we produced in the last chapter. However, due to time constraint and 
complexity of shuffle primitive, 
we ended up verifying a simple certification, which did not not involve any zero-knowledge-proof 
of shuffle.  Finally, in this chapter we closed the loop of decrease in number of scrutineers because 
any one can run the certificate checker. Moreover, we open sourced the checker, so 
that it can be inspected by anyone (we would like to call it correctness by democratic 
process).  One thing we would like to emphasize that cryptographic concepts are 
inherently very complex, so running a certificate checker certainly not amounts 
to understanding the various bits of cryptography and formal method used to develop the
certificate checker.

In the next chapter, we will discuss some of the properties of Schulze method which we have 
formalized in Coq. 











































   
   
   
   
   
   