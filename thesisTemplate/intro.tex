\chapter{Introduction}
\label{cha:intro}

\epigraph{The best weapon of a dictatorship is secrecy, but the best weapon of a democracy should be the weapon of openness.} 
{\textit{Niels Bohr}} 

%"People shouldn't be afraid of their government. Governments should be afraid of their people." - Alan Moore, V for Vendetta 
%
%"Those who cast the vote decides nothing. Those who count the vote decide everything." - Joseph Stalin


A democracy can be best describe as a system where every participant 
has equal right to express his opinion(s) on different matters. Each
eligible participant expresses his opinion, and a winner is elected 
by using some method to combine every participant's opinion.
In current context of democratic elections, 
every eligible participant expresses his opinion on a paper, also 
known as ballot, 
by marking a tick against the preferred candidate 
(first-past-the-post) or ranking the candidates in some order
(preferential voting). 
One of the most fundamental assumption during any democratic 
election, privacy and anonymity about the voter's choices, was not the case 
in past. It was Victoria, Australia who first introduce the 
concept of secret ballot in 1856 \footnote{
 https://trove.nla.gov.au/work/25532676?q\&versionId=30761831}
 \footnote{https://www.nma.gov.au/defining-moments/
	  resources/secret-ballot-introduced} and gradually, it 
	  got adoption all over the world. Before 1856, Victoria
	  and NSW held their elections to elect its 
	  democratic representative in pub where it was legal for 
	  candidates to offer beer to voters to influence their 
	  decision! 
	  
Verifiability or transparency is other fundamental assumption for 
any democratic election. Transparency insures that each step of 
of election process is easily understood and can be scrutinize by 
every stake holders e.g. voters, political parties, and external 
observers. Privacy and verifiability  is basic 
ingredient for coercion free election, i.e. people 
can express their opinion with any concern, and at the same time,
they can not sell their votes. Without privacy and verifiability, 
we can not expect a true democracy.   

 
 \begin{figure}[htb]
	\begin{center}
	\includegraphics[scale=0.25]{NorthBourke.jpg}
	\caption{Election held in 1855 in Victoria, Australia 
	  was conducted in pub!}
	\end{center}
  \end{figure}   
  
 Electronic voting is very different from paper ballot election. 
 Reasoning 
 about privacy and verifiability and many other properties 
 in paper ballot election 
 is straight forward, but we can not do the same for 
 electronic voting election. The reason is, electronic voting constitutes 
 of various untrusted component of software
 \footnote{Unix operating system has 15 million code which can not 
 trusted at all} and hardware
 \footnote{Intel has many undocumented instructions in its x86 
 processor. https://github.com/xoreaxeaxeax/sandsifter} which 
 exposes a large attack surface. These attack surfaces could 
 potentially be exploited for illegal gain in election. 
 Even though electronic voting is gaining popularity, none so far has 
 addressed the software bug issue in electronic voting context. 
 Most of the software used in electronic voting process are 
 inadequately tested which evidently can not rule out all 
 the possibilities.
  
 \section{Motivation and Research Statement}:
 Electronic voting has gained a lot of attention in recent year,  and  there are extensive work which 
 addresses the different issues related of electronic voting protocols  in symbolic model.
% , 
% but there are very few, to the best of my knowledge, 
% which has used theorem prover to implement the voting protocol (counting algorithm)
% and verify its correctness properties. 
 \cite{10.1007/978-3-540-31987-0_14}, and  \cite{Delaune2010} have used pi-calculus to model 
 and analyse various properties, fairness, eligibility, vote-privacy, receipt-freeness and coercion-resistant,  
 of the protocol FOO developed by \cite{10.1007/3-540-57220-1_66}.  \cite{Backes:2008:AVR:1380848.1381255}
 presented a general technique to model  remote electronic 
 voting protocol and automatically verifying  its security properties using pi-calculus. 
 \cite{5992139} have used pi-calculus to analyse the ballot secrecy of \cite{Helios:2016:HVS}.
 \cite{10.1007/978-3-642-28641-4_7} have used pi-calculus to ascertain the properties of 
 Norwegian electronic voting protocol.
 \cite{10.1007/978-3-319-68687-5_7} have used Tamarin  to prove receipt-freeness 
 and vote-privacy of Selene voting protocol \citep{Selene}.  Most of these work differs from ours
 in the sense that their primary focus is verification of security protocol in  
 Dolev-Yao model or  complexity-theoretic model, whereas our work is 
 more focused on verified implementation and  verifiability  aspect of election.
[Note to myself: Carsten has used the term "auditable correctness of implementations" for this]
 
 The closest to our work is \cite{DeYoung:2012:LLV}, \cite{Pattinson:2015:VCM}, \cite{Pattinson:2016:MSP},
 \cite{Verity:2017:FVI:3014812.3014845}, and \cite{Ghale:2017:FVS}.  \cite{DeYoung:2012:LLV} 
 used linear logic\citep{GIRARD19871} to model the different entity in electronic voting as a resource. 
 The use of linear logic makes it very natural to capture the different entities in electronic voting,  
 depending on their usage, by means of modality e.g. a voter can cast only one vote, but he might 
 need to show his photo id to multiple times at counting booth. \cite{Pattinson:2015:VCM} treated 
 the process of vote counting from
 the perspective of mathematical proof. They used (mathematical) proof theory to model the 
 counting. \cite{Ghale:2017:FVS} have formalized the single transferable votes in Coq and 
 extracted a Haskell code from the formalization. The extracted Haskell code produces the result 
 and a certificate for a given set of input ballots. The certificate can be used by any third party to verify 
 or audit the outcome of election result.  However, none of these work considers the encrypted ballots, 
 and they simply count on plaintext ballot which is susceptible to "italian attack" \citep{Benaloh:2009:SSC}. 
 This work not only considers the plaintext ballot, but extends the counting to encrypted ballot 
 using homomorphic encryption and uses the zero-knowledge-proof to make sure that outcome of 
election can be verified by any independent third party. 
 

After inspecting the current state of art in electronic voting, we 
 asked ourselves two questions:
 \begin{enumerate}
  \item Can we engineer a system which is formally verified to 
    guarantee the correctness property, and practical enough
    to count the real life election involving millions of ballots ? 
 \item  Can we decouple the verifiability from implementation i.e. 
    generating enough evidence so that any independent auditor can 
    ascertain the outcome of election without trusting the implementation 
    of software used to conduct the election ? 
  \item How "good"  is our proposed solution ?
  \end{enumerate}
  
  
 In order to answer these three questions, we needed three things:
\begin{enumerate}
  \item A voting protocol
  \item A theorem prover to implement the voting protocol
  \item Parameters, agreed upon properties in electronic voting research community, against which 
  		  we will measure our solution
\end{enumerate} 
 
 \subsection{Voting Protocol}
 For voting protocol, we settled down with \cite{Schulze:2011:NMC}.
 Even though Schulze method is not used in any democratic election, the reason 
 we went ahead with it because it is interesting  and at the same time, non-trivial. 
 Schulze's method elects  a single winner based on 
preferential votes.  At the same time, Arrow's impossibility theorem \citep{Arrow:1950:DCS} states
 that no preferential voting 
scheme can have all the desired properties established by  social choice theorist,
the Schulze's method offers a good balance. Many of these properties are already 
established in his original paper.  We will discuss more about Schulze method in 
\ref{cha:schulze_method}, and about its properties in \ref{cha:machine_checked}. 

\subsection{Theorem Prover:}
Our preference for theorem prover was Coq \citep{Bertot:2004:ITP}. The 
reason for choosing Coq is that it supports an expressive logic and  (crucial for us) dependent 
inductive types.  Coq has a well developed extraction facility that 
we use to extract proofs into OCaml programs, and using these extracted OCaml programs, we 
have counted the ballots from election to produce the result.  We will discuss more about the Coq in 
\ref{cha:background}. 

[Todo : Fill the details]

\subsection{Properties of Electronic Voting Protocol:} 
 What makes a electronic voting protocol  "good" or "desirable" ?  Some commonly sought 
 properties which a electronic voting protocol must have are given below \citep{5958051}, 
 \citep{Benaloh:1994:RSE:195058.195407},  \citep{Delaune:2010:VPT}, \citep{Bernhard:2017:PES}.
 \begin{itemize}
 
  \item Correctness:
 	The produced results are correct, and convincing to all leaving no  ground for suspicion. 
 	
 \item Privacy:
    All the votes must be secret, and voter should not be able to convince anyone the 
    value of his vote.
 
 \item End-to-end Verifiability:
 
 \begin{itemize}
  \item Cast-as-intended: Every voter can verify that their ballot was cast as
  intended
  \item Collected-as-cast: Every voter can verify that their ballot was collected as
  cast
  \item Tallied-as-cast: Everyone can verify final result on the basis of the
  collected ballots.
\end{itemize}

\item Practicality:
  
\item Coercion-resistance:
	A voter cannot cooperate with a coercer to prove to him that she voted in a certain way.
  

\item Voter Eligibility:
  Only eligible voter can cast the ballot, and only once.

\item Fairness:
  No intermediate results can be obtained which could influence the remain-ing voters

\item Receipt-freeness:
A voter does not gain any information (receipt) which can be used to prove to a coercer that
she voted in a certain way

 \end{itemize}
 
  
 
% 
%	\begin{itemize}
%	\item Correctness : The produced results are correct, and convincing to all leaving no 
%	          ground for suspicion. 
%	\item Coercion-resistance: 
%	\item Eligibility
%	\item Fairness
%	\item Privacy
%	\item Practicality
%	\item Individual-verifiability
%	\item Universal-verifiability
%	\item Receipt-freeness
%	\end{itemize}


The focus of our study is the \textit{Correctness}, 
 \textit{Practicality}, \textit{Verifiability}, i.e. generating enough evidence so that 
 everyone can verify final result on the basis of the collected ballots, and \textit{Privacy}. In the critical 
 summary of each chapter, we will measure our solution against these properties. 
 
 
%
%\textbf{Chapter Overview}
%[Leave it till the end. Do I need chapter overview for this chapter ?]


%\section{Motivation and Problem Statement}
% \fix{What is the motivation behind this research ? Clearly state your \
%      Problem statement}
	      
%  Before I start diving deep into explaining the bits and pieces 
%  of electronic voting, Schulze method, and Coq theorem prover, 
%  I still need to persuade the reader that why this thesis 
%  exists in first place ?  Why did I choose to certify Schulze 
%  algorithm in Coq given the fact that it is not used in 
%  any democratic election and Coq is not serious business
%  in certifying the software used in electronic voting.  
%  Probably, this is the only time where I would unearth 
%  the philosopher inside me and go to length to explain 
%  the implication of my work. 
%  
  
       
%  Well, the honest answer is that  I want to graduate (hopefully), but
%  it's still not convincing argument because I could have chosen 
%  some other project and graduate. 
%%   
%  [Motivate should refer to some past research, but none of 
%  them address the problem of bugs in software.  
%  
%  On the serious note, the motivation
%  behind this thesis to find the answer of question:  
%  
%  Can we afford bugs in software used for vote counting, and 
%  can we engineer a system which has all the "desirable properties" ?
%  
%  We would answer the 
%  
    
      
 
 

\section{Contribution}
 	

	\subsection{Publication}
	

\section{Outline of the Chapters}
[Rewrite again when you finish everything]
In chapter 2, we will give a brief history of electronic voting, the glitches that cause some countries 
to withdraw from electronic voting.  



We will evaluate our work based on the agreed upon properties of electronic voting in 
the research community.  It is privacy, verifiability and practicality. In the end of each chapter, 
we will give a critical summary of  our work and which parameter is passed successfully and which 
one it failed. 



%
%Given that I am 
%computer scientist by profession, I would not try to justify my decisions by excessive use of philosophical 
%arguments, but at this point it seems very apt to first investigate this question from philosophical point. 
%
%
%
%The answer depends on how you perceive democracy.  For a dictator, probably  bug in the 
%vote software would be a natural choice, among many others,  to rig the election. If you firmly believe 
%in democracy and democratic values, then 
%among many other things, transparency and bug freeness in vote counting software  would also be 
%in your agenda. There is no doubt that technology can play a critical role in maintaining the democratic values, 
%but assuming that it is the only factor would be a gross mistake. 
%In Azerbaijan's 2013 election, the running president Ilham Aliyev launched a iPhone app, to boost the 
%credibility of election, which enabled the citizens of Azerbaijan to 
%track the tallies as counting took place. There was just one problem. The app already showed that 
%Ilham Aliyev elected before even a ballot was counted.  In this particular case, technology merely helped in 
%surfacing the problem, but it did not do any other thing.  More often technology can be used to hide 
%the transparency of system than making it evident specially in corrupt society for personal gain. 
%
%
%Democracy is a complex system of different actors interacting with each other in certain fashion.  How to make 
%these interaction more productive and better for society, I leave this to political scientists and social scientist
%to figure out, and I stick to my job as a technology enabler.  This thesis is  my journey 
%(with my supervisor) about finding  a way to make vote counting software more robust and transparent.


%
%\section{Why Schulze ?}
%\label{sec:thesisstatement}
%Even though Schulze method is not used in any democratic election, we settled down 
%with it because it is interesting  and at the same time, non-trivial. 
% Schulze's method [cite Schulze]  elects  a single winner based on 
%preferential votes.  At the same time, Arrow's impossibility theorem [cite Arrow]  states that no preferential voting 
%scheme can have all the desired properties established by  social choice theorist,
%the Schulze's method offers a good balance. Many of these properties are already 
%established in his original paper. These properties are Non-dictatorship,  Pareto,  Monotonicity, 
%Resolvability, Independence of Clones 
%
%I will discuss some of its properties in next chapter. 
%
%
%A  quantitative  comparison of voting methods 
%[cite An Optimal Single-Winner Preferential Voting System Based onGame Theory]  also shows that 
%Schulze voting is better (in a game theoretic sense) than other, more established, systems.  The 
%Schulze Method is rapidly gaining popularity in the open software community, and It is one of the most 
%popular voting protocol over internet to elect candidates. At of 22 July 2019, Wikipedia entry on Schulze 
%method [cite Wiki entry] shows at least 70 users, and some of 
%notable users among them are Gentoo Foundation, Debian, GNU Privacy Guard (GnuPG), Ubuntu, and 
%Pirate Party in Australia, Austria, Belgium, Brazil, Germany, Iceland, Italy, 
%Netherlands, New Zealand, Sweden, Switzerland, and United States.  
%
%
%
%
%
%
%\section{Why Coq}
\label{sec:outline}
How many chapters you have? You may have Chapter~\ref{cha:background},
Chapter~\ref{cha:design}, Chapter~\ref{cha:methodology},
Chapter~\ref{cha:result}, and Chapter~\ref{cha:conc}.
