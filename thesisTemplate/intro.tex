\chapter{Introduction}
\label{cha:intro}

\epigraph{The best weapon of a dictatorship is secrecy, but the best weapon of a democracy should be the weapon of openness.} 
{\textit{Niels Bohr}} 

\section{Problem Statement}
	Electronic voting is getting popular in many countries, and the reason for popularity is 
	cost-effectiveness, faster results, and high voter turnouts. India has 900 million eligible voter, 
   and with 67 percent (roughly 600 million) voter turnout in 2019 general election, 
   the result was declared in 2 days. This was possible because India uses  electronic 
   voting machines (EVM) to conduct the election. Australia has 
   compulsory voting, but its massive size with sparsely populated land makes 
   the election a big logistic challenge.  Australia is actively pursuing Internet voting 
   to ease the logistic challenges and engage more citizens from the remote places. Estonia
   is praised for successfully implementing the Internet voting since 2005. The first country to
   adopt internet voting and showing confidence in technology. In 2019 elections, 
   as claimed on e-estonia \citep{Estonia},  the time saved was 11,000 working days. 
   
   Despite all these benefits, electronic voting is nightmare because the minuscule possibility of 
   a bug in software used in voting could lead to a disaster, possibly 
   inverting the results \citep{TSwiss},
   \citep{10.1007/978-3-319-22270-7_3}, \citep{ARANHA2019335},
   \citep{Feldman:2007:SAD:1323111.1323113}. 
   By its inherent nature electronic voting has many 
   problems, which are not present in paper ballot elections that makes it perfectly susceptible 
   to delivering wrong and unverifiable result  \citep{Wolchok:2010:SAI:1866307.1866309}.
   The software and hardware used in the electronic voting process  
	are treated as a black-box and commercial in confidence \citep{AEC:2013:LMM} which  
	violates the fundamental property of  public examinability of any democratic election. 
	More often than not, these software programs are configured incorrectly \citep{1301313} and 
	run at top of (untrusted) operating system and hardware.  Operating systems have
	millions of lines of code (Linux has 15 million code) which exposes a large attack surface 
	and could be  exploited for illegal gain in election, possibly by 
	current government or foreign country.  As a consequence,  from casting ballot electronically to declaring 
	results based on electronic data (ballot) may raise several questions:
	
	\begin{enumerate}
   \item How do we know the ballot we have cast according to our intent is successfully recorded 
   without any change by the software or hardware used in the process?
   \item How do we know that the software or hardware used to conduct the election is not 
   behaving maliciously and not  changing my intent? 
   \item How do we know that the software used in counting process has produced the correct result, and it
   is based on the ballots cast by eligible voters, and it has not added  some fake ballots to favour 
   some candidate?
   \item How do we know that software reading the ballot from ballot-storing-hardware 
    is indeed the ballot we cast, and hardware has not changed it? 
   \item How do we know that the software used in counting process has no bugs, or the hardware is 
    not malfunctioning? 
    \item How do we know that the software used in ballot cast process is not linking my credentials 
    against my ballot (choices)?
   \end{enumerate}
   
   
   Elections conducted  electronically, electronic voting, lacks basic assumption of any democratic 
	election, i.e. correctness, privacy, and verifiability.
	These problems are well recognized in electronic voting community, 
   and some commonly sought 
   properties which a electronic voting protocol must have are 
    \citep{5958051}, 
   \citep{Benaloh:1994:RSE:195058.195407},  \citep{Delaune:2010:VPT}, \citep{Bernhard:2017:PES}:
 \begin{itemize}
 
  \item Correctness:
 	The produced results are correct, and convincing to all leaving no  ground for suspicion. 
 	
 \item Privacy:
    All the votes must be secret, and voter should not be able to convince anyone the 
    value of his vote.
 
 \item End-to-end Verifiability:
 Any independent third party should be able to verify the final outcome of election based on cast 
 ballots.  It can be further divided into three sub-category:
 
 \begin{itemize}
  \item Cast-as-intended: Every voter can verify that their ballot was cast as
  intended
  \item Collected-as-cast: Every voter can verify that their ballot was collected as
  cast
  \item Tallied-as-cast: Everyone can verify final result on the basis of the
  collected ballots.
\end{itemize}
\end{itemize}
	

In this thesis, we focus on privacy, correctness, and tallied-as-cast, the third part of end-to-end verifiability.
We assume the first two properties of end-to-end verifiability, cast-as-intended and collected as cast, hold
for the ballots published on bulletin board. 

\section{Research Motivation and Contribution}
Given the potential advantages of electronic voting,  we need to address
the correctness, privacy and verifiability concerns for its widespread adoption. 
This thesis sets out to address these concerns of electronic voting. 
The questions we asked ourself was:
 \begin{enumerate} 
  \item Can we implement a vote counting protocol with the  
    guaranteed correctness of its properties, and practical enough
    to count the real life election involving millions of ballots (Correctness)?
  \item Can we produce the result by counting encrypted ballot without revealing 
  its content, and at the same time, 
  assuring everyone that the result produced is only based on valid ballots, 
  and invalid ones have been discarded  (Privacy)?
 \item Can we decouple the verifiability from implementation, i.e. 
    generating enough evidence so that any independent auditor can 
    ascertain the outcome of election without trusting the implementation 
    of software used to conduct the election (Verifiability)?
  \end{enumerate}

\noindent
We answer these question by taking the Schulze method \citep{Schulze:2011:NMC} 
as an example and Coq \citep{Bertot:2004:ITP}
theorem prover  for implementing and proving the correctness of  Schulze method.
Even though Schulze method is not used in any democratic election to public office, the reason 
 we went ahead with it because it has many interesting properties and, 
 at the same time, it is non-trivial.   Schulze's method elects  a single winner based on 
preferential votes.  Arrow's impossibility theorem \citep{Arrow:1950:DCS} states
 that no preferential voting 
scheme can have all the desired properties established by  social choice theorist,
the Schulze's method offers a good balance. We will discuss more about the Schulze method in 
chapter \ref{cha:schulze_method}, and its properties in chapter \ref{cha:machine_checked}. 
The reason for choosing Coq is that it supports an expressive logic and dependent 
inductive types which is very crucial for us. We will discuss more about the Coq in 
chapter \ref{cha:theorem_crypto}.

We demonstrate the:
\begin{enumerate}
 \item \texttt{Correctness} by formally specifying the Schulze method  inside 
 Coq theorem prover, and prove the correctness properties. 
 Coq has a well developed extraction facility that 
 we use to extract proofs into OCaml programs, and using these extracted OCaml programs, we 
 have counted the ballots from election to produce the result. 
 \item \texttt{Verifiability} by tabulating the relevant data of election. We call it scrutiny-sheet/certificate. 
   Achieving verifiability in a plain-text ballot counting is fairly straight forward, but it is not 
   the same with encrypted ballot counting.  To achieve verifiability in encrypted ballot counting, 
   we augment the scrutiny sheet with zero-knowledge-proof for the each claim we make during the 
   counting which can  later be checked by any auditor.  
 \item \texttt{Privacy} by encryption. We use homomorphic-encryption to compute the 
  finally tally without decrypting any individual ballot. 
\end{enumerate}


In addition to this, we have also developed a formally verified certificate checker to ease the 
auditing of election conducted on encrypted ballot.  Given that our certificate is very complex 
and formalizing all primitives involved would be fairly time consuming, we have developed a 
formally verified certificate checker for IACR 2018 (International Association of Cryptographic Research) election, 
relatively simple than ours, scrutiny sheet
(chapter \ref{cha:software_independence}). 
Also, we have proved couple of properties, Condercet winner, and Reversal symmetry property 
of the Schulze method inside Coq theorem prover (chapter \ref{cha:machine_checked}). 

\section{Cryptographic Blackbox}
The goal of this thesis is not to verify the cryptographic primitives, but use them as a 
facilitator to achieve privacy and verifiability in electronic voting. To achieve it, we have 
taken the axiomatic approach, and assumed the existence of cryptographic primitives 
inside Coq theorem prover with the 
axioms about their correctness behaviour. These primitives, in general, provide functionality 
of encrypting a plain-text, decrypting a cipher-text, constructing a zero-knowledge-proof, 
and verifying a zero-knowledge-proof. Later, in extracted OCaml, these functions are instantiated 
with Unicrypt\citep{LocherH14} function.  We will discuss more on this in chapter
\ref{cha:homormorphic_schulze}.
\footnote{Formalizing the whole cryptographic stack used in our 
project would be very time consuming (probably a PhD itself), but it would be worth trying. 
Although, we have formalized the (El-Gamal) encryption, and decryption inside Coq, but we still 
are very far from achieving the goal of fully verified cryptographic stack.  We leave the formalisation 
of cryptographic primitives for future work (work in progress).}



\section{Publication}
 The chapters, or some part of it,  of this thesis are based on the following papers:
	\begin{enumerate}
	\item Pattinson, D. and Tiwari, M., 2017. Schulze Voting as Evidence carrying computation. In Proc. 
	ITP 2017, vol. 10499 of Lecture Notes in Computer Science, 410–426. Springer. 
	\item Lyria Bennett Moses, Rajeev Goré, Ron Levy, Dirk Pattinson, Mukesh Tiwari:
	No More Excuses: Automated Synthesis of Practical and Verifiable Vote-Counting Programs for Complex 
	Voting 	Schemes. E-VOTE-ID 2017: 66-83
	\item Milad K. Ghale, Rajeev Goré, Dirk Pattinson, Mukesh Tiwari:
	Modular Formalisation and Verification of STV Algorithms. E-Vote-ID 2018: 51-66
	\item Verifiable Homomorphic Tallying for the 
 		Schulze Vote Counting Scheme (VSSTE paper)
	\item Verified Verifiers for Verifying Elections (CCS paper)
	\end{enumerate}
 \noindent
 Part of chapter \ref{cha:background} is based on \textit{No More Excuses: Automated Synthesis of Practical 
 and Verifiable Vote-Counting Programs for Complex Voting  Schemes},
 chapter \ref{cha:schulze_method} is based on \textit{Schulze Voting as Evidence Carrying Computation},
 chapter \ref{cha:homormorphic_schulze} is based on \textit{Verifiable Homomorphic Tallying for the 
 Schulze Vote Counting Scheme}, and chapter
 \ref{cha:software_independence} is based on \textit{ Verified Verifiers for Verifying Elections}.




\section{Related Work}
 There is extensive work which 
 addresses the different issues related of electronic voting protocols  in symbolic model, 
 but there are very few, to the best of my knowledge, 
 that have used theorem provers to implement the voting protocol (counting algorithm)
 and verify its correctness properties. 
 \cite{10.1007/978-3-540-31987-0_14}, and  \cite{Delaune2010} have used pi-calculus to model 
 and analyse various properties, such as fairness, eligibility, vote-privacy, receipt-freeness and 
 coercion-resistant,  
 of the protocol FOO developed by \cite{10.1007/3-540-57220-1_66}.  \cite{Backes:2008:AVR:1380848.1381255}
 presented a general technique to model  remote electronic 
 voting protocol and automatically verifying  its security properties using pi-calculus. 
 \cite{5992139} have used pi-calculus to analyse the ballot secrecy of \cite{Helios:2016:HVS}.
 \cite{10.1007/978-3-642-28641-4_7} have used pi-calculus to ascertain the properties of 
 Norwegian electronic voting protocol.
 \cite{10.1007/978-3-319-68687-5_7} have used Tamarin  to prove receipt-freeness 
 and vote-privacy of Selene voting protocol \citep{Selene}.  Most of these work differs from ours
 in the sense that their primary focus is verification of security protocol in  
 Dolev-Yao model or  complexity-theoretic model, whereas our work is 
 more focused on verified implementation and  verifiability  aspect of vote counting.

 The closest to our work is \cite{DeYoung:2012:LLV}, \cite{Pattinson:2015:VCM}, \cite{Pattinson:2016:MSP},
 \cite{Verity:2017:FVI:3014812.3014845}, and \cite{Ghale:2017:FVS}.  \cite{DeYoung:2012:LLV} 
 used linear logic\citep{GIRARD19871} to model the different entity in electronic voting as a resource. 
 The use of linear logic makes it very natural to capture the different entities in electronic voting,  
 depending on their usage, by means of modality e.g. a voter can cast only one vote, but he might 
 need to show his photo id to multiple times at counting booth. \cite{Pattinson:2015:VCM} treated 
 the process of vote counting from
 the perspective of mathematical proof. They used (mathematical) proof theory to model the 
 counting. \cite{Ghale:2017:FVS} have formalized the single transferable votes in Coq and 
 extracted a Haskell code from the formalization. The extracted Haskell code produces the result 
 and a certificate for a given set of input ballots. The certificate can be used by any third party to verify 
 or audit the outcome of election result.  However, none of these work considers privacy as a key 
 issue in electronic voting, and their method simply works for plaintext ballots which are  susceptible to 
 "italian attack"  \citep{Otten}   \citep{Benaloh:2009:SSC}.

\section{Outline of the Chapters}
[Rewrite when you are done with this thesis]
Chapter \ref{cha:background} provides an overview of electronic voting around the world, 
problems in general, and rationale for formal verification of election voting software. 
Chapter \ref{cha:theorem_crypto} provides the overview of concept of 
Coq theorem prover  and cryptographic primitives. Chapter \ref{cha:schulze_method} 
describes Schulze method, its formal specification, proof of correctness, experimental result, 
and scrutiny sheet.  
Chapter \ref{cha:homormorphic_schulze} describes 
verifiable homomorphic tally for Schulze method, its realization in theorem prover, experimental 
result,  instructions to audit the scrutiny sheet. 
Chapter \ref{cha:software_independence} focuses on the notion of software independence, and 
formalization of  
Sigma protocol. Chapter \ref{cha:machine_checked} (ongoing work) describes the 
properties of Schulze method. 
[Todo : Rewrite the last line when you have last chapter. Hopefully, that would be last 
day of writing :)]
Finally, chapter 8 concludes the thesis, and some possible direction of future work. 



\section{Trivia}
 Before 1856, Victoria and NSW held their elections to elect its 
	  democratic representative in pub where it was legal for 
	  candidates to offer beer to voters to influence their 
	  decision! 
	  
	   \begin{figure}[htb]
	\begin{center}
	\includegraphics[scale=0.25]{NorthBourke.jpg}
	\caption{Election held in 1855 in Victoria, Australia 
	  was conducted in pub!}
	\end{center}
  \end{figure}   
  
  



%
%
%\section{Delete everything Below}
%
%
%A democracy can be best describe as a system where every participant 
%has equal right to express his opinion(s) on different matters. Each
%eligible participant expresses his opinion, and a winner is elected 
%by using some method to combine every participant's opinion.
%In current context of democratic elections, 
%every eligible participant expresses his opinion on a paper, also 
%known as ballot, 
%by marking a tick against the preferred candidate 
%(first-past-the-post) or ranking the candidates in some order
%(preferential voting). 
%One of the most fundamental assumption during any democratic 
%election, privacy and anonymity about the voter's choices, was not the case 
%in past. It was Victoria, Australia who first introduce the 
%concept of secret ballot in 1856 \footnote{
% https://trove.nla.gov.au/work/25532676?q\&versionId=30761831}
% \footnote{https://www.nma.gov.au/defining-moments/
%	  resources/secret-ballot-introduced} and gradually, it 
%	  got adoption all over the world. Before 1856, Victoria
%	  and NSW held their elections to elect its 
%	  democratic representative in pub where it was legal for 
%	  candidates to offer beer to voters to influence their 
%	  decision! 
%	  
%Verifiability or transparency is other fundamental assumption for 
%any democratic election. Transparency insures that each step of 
%of election process is easily understood and can be scrutinize by 
%every stake holders e.g. voters, political parties, and external 
%observers. Privacy and verifiability  are basic 
%ingredient for coercion free election, i.e. people 
%can express their opinion with any concern, and at the same time,
%they can not sell their votes. Without privacy and verifiability, 
%we can not expect a true democracy.   
%
% 
% \begin{figure}[htb]
%	\begin{center}
%	\includegraphics[scale=0.25]{NorthBourke.jpg}
%	\caption{Election held in 1855 in Victoria, Australia 
%	  was conducted in pub!}
%	\end{center}
%  \end{figure}   
%  
% Electronic voting is very different from paper ballot election. 
% Reasoning 
% about privacy and verifiability and many other properties 
% in paper ballot election 
% is straight forward, but we can not do the same for 
% electronic voting election. The reason is, electronic voting constitutes 
% of various untrusted component of software
% \footnote{Unix operating system has 15 million code which can not 
% trusted at all} and hardware
% \footnote{Intel has many undocumented instructions in its x86 
% processor. https://github.com/xoreaxeaxeax/sandsifter} which 
% exposes a large attack surface. These attack surfaces could 
% potentially be exploited for illegal gain in election. 
% Even though electronic voting is gaining popularity, none so far has 
% addressed the software bug issue in electronic voting context. 
% Most of the software used in electronic voting process are 
% inadequately tested which evidently can not rule out all 
% the possibilities.
%  
% \section{Motivation and Research Statement}
% \textbf{Motivation:}
% Electronic voting has gained a lot of attention in recent year,  and  there are extensive work which 
% addresses the different issues related of electronic voting protocols  in symbolic model.
%% , 
%% but there are very few, to the best of my knowledge, 
%% which has used theorem prover to implement the voting protocol (counting algorithm)
%% and verify its correctness properties. 
% \cite{10.1007/978-3-540-31987-0_14}, and  \cite{Delaune2010} have used pi-calculus to model 
% and analyse various properties, fairness, eligibility, vote-privacy, receipt-freeness and coercion-resistant,  
% of the protocol FOO developed by \cite{10.1007/3-540-57220-1_66}.  \cite{Backes:2008:AVR:1380848.1381255}
% presented a general technique to model  remote electronic 
% voting protocol and automatically verifying  its security properties using pi-calculus. 
% \cite{5992139} have used pi-calculus to analyse the ballot secrecy of \cite{Helios:2016:HVS}.
% \cite{10.1007/978-3-642-28641-4_7} have used pi-calculus to ascertain the properties of 
% Norwegian electronic voting protocol.
% \cite{10.1007/978-3-319-68687-5_7} have used Tamarin  to prove receipt-freeness 
% and vote-privacy of Selene voting protocol \citep{Selene}.  Most of these work differs from ours
% in the sense that their primary focus is verification of security protocol in  
% Dolev-Yao model or  complexity-theoretic model, whereas our work is 
% more focused on verified implementation and  verifiability  aspect of election.
%[Note to myself: Carsten has used the term "auditable correctness of implementations" for this]
% 
% The closest to our work is \cite{DeYoung:2012:LLV}, \cite{Pattinson:2015:VCM}, \cite{Pattinson:2016:MSP},
% \cite{Verity:2017:FVI:3014812.3014845}, and \cite{Ghale:2017:FVS}.  \cite{DeYoung:2012:LLV} 
% used linear logic\citep{GIRARD19871} to model the different entity in electronic voting as a resource. 
% The use of linear logic makes it very natural to capture the different entities in electronic voting,  
% depending on their usage, by means of modality e.g. a voter can cast only one vote, but he might 
% need to show his photo id to multiple times at counting booth. \cite{Pattinson:2015:VCM} treated 
% the process of vote counting from
% the perspective of mathematical proof. They used (mathematical) proof theory to model the 
% counting. \cite{Ghale:2017:FVS} have formalized the single transferable votes in Coq and 
% extracted a Haskell code from the formalization. The extracted Haskell code produces the result 
% and a certificate for a given set of input ballots. The certificate can be used by any third party to verify 
% or audit the outcome of election result.  However, none of these work considers privacy as a key 
% issue in electronic voting, and their method simply works for plaintext ballots which are  susceptible to 
% "italian attack"  \citep{Otten}   \citep{Benaloh:2009:SSC}.
% 
% This work not only considers the plaintext ballot, but extends the counting to encrypted ballot 
% using homomorphic encryption to preserve the privacy of election. To ensure the verifiability, 
% we use the zero-knowledge-proof to 
% make sure that outcome of election can be attested by any independent third party. To ease the 
% auditing or certificate-checking of election conducted on encrypted ballot with other complex 
% cryptographic entities, we have developed 
% a formally verified certificate checker for auditing the election \footnote{IACR 2018 election}. 
% To the best of our knowledge, no one has developed a vote counting system which counts encrypted 
% ballots, and at the same time, the implementation itself is verified inside a theorem prover.  
% \citep{10.1007/978-3-030-03592-1_5} has developed a verified certificate checker, but, again, 
% it involves plaintext ballots  with simple mathematics of addition and multiplication while  our checker
% tackles the certificates involving complex cryptographic 
% objects, e.g. encrypted ballots and zero-knowledge-proof of different statements.
% 
% 
%\textbf{Research Statement:}
%After inspecting the current state of art in electronic voting, we 
% asked ourselves three questions:
% \begin{enumerate}
%  \item Can we engineer a system which is formally verified to 
%    guarantee the correctness property, and practical enough
%    to count the real life election involving millions of ballots ? 
% \item  Can we decouple the verifiability from implementation, i.e. 
%    generating enough evidence so that any independent auditor can 
%    ascertain the outcome of election without trusting the implementation 
%    of software used to conduct the election ? 
%  \item How "good"  is our proposed solution ?
%  \end{enumerate}
%  
%  
% In order to answer these three questions, we needed three things:
%\begin{enumerate}
%  \item A voting protocol
%  \item A theorem prover to implement the voting protocol
%  \item Parameters, agreed upon properties in electronic voting research community, against which 
%  		  we will measure our solution
%\end{enumerate} 
% 
% \subsection{Voting Protocol}
% For voting protocol, we settled down with \cite{Schulze:2011:NMC}.
% Even though Schulze method is not used in any democratic election, the reason 
% we went ahead with it because it is interesting  and at the same time, non-trivial. 
% Schulze's method elects  a single winner based on 
%preferential votes.  At the same time, Arrow's impossibility theorem \citep{Arrow:1950:DCS} states
% that no preferential voting 
%scheme can have all the desired properties established by  social choice theorist,
%the Schulze's method offers a good balance. Many of these properties are already 
%established in his original paper.  We will discuss more about Schulze method in 
%\ref{cha:schulze_method}, and about its properties in \ref{cha:machine_checked}. 
%
%\subsection{Theorem Prover:}
%Our preference for theorem prover was Coq \citep{Bertot:2004:ITP}. The 
%reason for choosing Coq is that it supports an expressive logic and  (crucial for us) dependent 
%inductive types.  Coq has a well developed extraction facility that 
%we use to extract proofs into OCaml programs, and using these extracted OCaml programs, we 
%have counted the ballots from election to produce the result.  We will discuss more about the Coq in 
%\ref{cha:background}. 
%
%\subsection{Properties of Electronic Voting Protocol:} 
% What makes a electronic voting protocol  "good" or "desirable" ?  Some commonly sought 
% properties which a electronic voting protocol must have are given below \citep{5958051}, 
% \citep{Benaloh:1994:RSE:195058.195407},  \citep{Delaune:2010:VPT}, \citep{Bernhard:2017:PES}.
% \begin{itemize}
% 
%  \item Correctness:
% 	The produced results are correct, and convincing to all leaving no  ground for suspicion. 
% 	
% \item Privacy:
%    All the votes must be secret, and voter should not be able to convince anyone the 
%    value of his vote.
% 
% \item End-to-end Verifiability:
% Any independent third party should be able to verify the final outcome of election based on cast 
% ballots.  It can be further divided into three sub-category:
% 
% \begin{itemize}
%  \item Cast-as-intended: Every voter can verify that their ballot was cast as
%  intended
%  \item Collected-as-cast: Every voter can verify that their ballot was collected as
%  cast
%  \item Tallied-as-cast: Everyone can verify final result on the basis of the
%  collected ballots.
%\end{itemize}
%
%\item Practicality/Usability: 
%  
%\item Coercion-resistance:
%	A voter cannot cooperate with a coercer to prove to him that she voted in a certain way.
%  
%\item Voter Eligibility:
%  Only eligible voter can cast the ballot, and only once.
%
%\item Fairness:
%  No intermediate results can be obtained which could influence the remaining voters.
%
%\item Receipt-freeness:
%A voter does not gain any information (receipt) which can be used to prove to a coercer that
%she voted in a certain way
%
% \end{itemize}
% 
%  
% 
%% 
%%	\begin{itemize}
%%	\item Correctness : The produced results are correct, and convincing to all leaving no 
%%	          ground for suspicion. 
%%	\item Coercion-resistance: 
%%	\item Eligibility
%%	\item Fairness
%%	\item Privacy
%%	\item Practicality
%%	\item Individual-verifiability
%%	\item Universal-verifiability
%%	\item Receipt-freeness
%%	\end{itemize}
%
%
%The focus of our study is the \textit{Correctness}, 
% \textit{Practicality}, \textit{Verifiability}, i.e. generating enough evidence so that 
% everyone can verify final result on the basis of the collected ballots, and \textit{Privacy}. In the critical 
% summary of each chapter, we will measure our solution against these properties. 
% 
% 
%%
%%\textbf{Chapter Overview}
%%[Leave it till the end. Do I need chapter overview for this chapter ?]
%
%
%%\section{Motivation and Problem Statement}
%% \fix{What is the motivation behind this research ? Clearly state your \
%%      Problem statement}
%	      
%%  Before I start diving deep into explaining the bits and pieces 
%%  of electronic voting, Schulze method, and Coq theorem prover, 
%%  I still need to persuade the reader that why this thesis 
%%  exists in first place ?  Why did I choose to certify Schulze 
%%  algorithm in Coq given the fact that it is not used in 
%%  any democratic election and Coq is not serious business
%%  in certifying the software used in electronic voting.  
%%  Probably, this is the only time where I would unearth 
%%  the philosopher inside me and go to length to explain 
%%  the implication of my work. 
%%  
%  
%       
%%  Well, the honest answer is that  I want to graduate (hopefully), but
%%  it's still not convincing argument because I could have chosen 
%%  some other project and graduate. 
%%%   
%%  [Motivate should refer to some past research, but none of 
%%  them address the problem of bugs in software.  
%%  
%%  On the serious note, the motivation
%%  behind this thesis to find the answer of question:  
%%  
%%  Can we afford bugs in software used for vote counting, and 
%%  can we engineer a system which has all the "desirable properties" ?
%%  
%%  We would answer the 
%%  
%    
%      
% 
% 
%%
%%\section{Contribution}
%%  
%
%	\section{Publication}
%	During the course of PhD, I was fortunate to work with many researchers including my supervisor 
%	Dr. Dirk Pattinson, co-supervisor Rajeev Goré, colleague Milad K. Ghale,  and external collaborators 
%	Dr. Thomas Haines from NTNU, Dr. Lyria Bennett Moses from UNSW, and Dr Ron Levy from ANU.  
%	\begin{enumerate}
%	\item Pattinson, D. and Tiwari, M., 2017. Schulze voting as evidence carrying computation. In Proc. 
%	ITP 2017, vol. 10499 of Lecture Notes in Computer Science, 410–426. Springer. 
%	\item Lyria Bennett Moses, Rajeev Goré, Ron Levy, Dirk Pattinson, Mukesh Tiwari:
%	No More Excuses: Automated Synthesis of Practical and Verifiable Vote-Counting Programs for Complex 
%	Voting 	Schemes. E-VOTE-ID 2017: 66-83
%	\item Milad K. Ghale, Rajeev Goré, Dirk Pattinson, Mukesh Tiwari:
%	Modular Formalisation and Verification of STV Algorithms. E-Vote-ID 2018: 51-66
%	\item VSSTE paper
%	\item CCS paper
%
%	\end{enumerate}
%		
%	
%
%\section{Outline of the Chapters}
%[Rewrite again when you finish everything]
%In chapter 2, we will give a brief history of electronic voting, the glitches that cause some countries 
%to withdraw from electronic voting.  
%
%
%
%We will evaluate our work based on the agreed upon properties of electronic voting in 
%the research community.  It is privacy, verifiability and practicality. In the end of each chapter, 
%we will give a critical summary of  our work and which parameter is passed successfully and which 
%one it failed. 
%
%
%
%%
%%Given that I am 
%%computer scientist by profession, I would not try to justify my decisions by excessive use of philosophical 
%%arguments, but at this point it seems very apt to first investigate this question from philosophical point. 
%%
%%
%%
%%The answer depends on how you perceive democracy.  For a dictator, probably  bug in the 
%%vote software would be a natural choice, among many others,  to rig the election. If you firmly believe 
%%in democracy and democratic values, then 
%%among many other things, transparency and bug freeness in vote counting software  would also be 
%%in your agenda. There is no doubt that technology can play a critical role in maintaining the democratic values, 
%%but assuming that it is the only factor would be a gross mistake. 
%%In Azerbaijan's 2013 election, the running president Ilham Aliyev launched a iPhone app, to boost the 
%%credibility of election, which enabled the citizens of Azerbaijan to 
%%track the tallies as counting took place. There was just one problem. The app already showed that 
%%Ilham Aliyev elected before even a ballot was counted.  In this particular case, technology merely helped in 
%%surfacing the problem, but it did not do any other thing.  More often technology can be used to hide 
%%the transparency of system than making it evident specially in corrupt society for personal gain. 
%%
%%
%%Democracy is a complex system of different actors interacting with each other in certain fashion.  How to make 
%%these interaction more productive and better for society, I leave this to political scientists and social scientist
%%to figure out, and I stick to my job as a technology enabler.  This thesis is  my journey 
%%(with my supervisor) about finding  a way to make vote counting software more robust and transparent.
%
%
%%
%%\section{Why Schulze ?}
%%\label{sec:thesisstatement}
%%Even though Schulze method is not used in any democratic election, we settled down 
%%with it because it is interesting  and at the same time, non-trivial. 
%% Schulze's method [cite Schulze]  elects  a single winner based on 
%%preferential votes.  At the same time, Arrow's impossibility theorem [cite Arrow]  states that no preferential voting 
%%scheme can have all the desired properties established by  social choice theorist,
%%the Schulze's method offers a good balance. Many of these properties are already 
%%established in his original paper. These properties are Non-dictatorship,  Pareto,  Monotonicity, 
%%Resolvability, Independence of Clones 
%%
%%I will discuss some of its properties in next chapter. 
%%
%%
%%A  quantitative  comparison of voting methods 
%%[cite An Optimal Single-Winner Preferential Voting System Based onGame Theory]  also shows that 
%%Schulze voting is better (in a game theoretic sense) than other, more established, systems.  The 
%%Schulze Method is rapidly gaining popularity in the open software community, and It is one of the most 
%%popular voting protocol over internet to elect candidates. At of 22 July 2019, Wikipedia entry on Schulze 
%%method [cite Wiki entry] shows at least 70 users, and some of 
%%notable users among them are Gentoo Foundation, Debian, GNU Privacy Guard (GnuPG), Ubuntu, and 
%%Pirate Party in Australia, Austria, Belgium, Brazil, Germany, Iceland, Italy, 
%%Netherlands, New Zealand, Sweden, Switzerland, and United States.  
%%
%%
%%
%%
%%
%%
%%\section{Why Coq}
%\label{sec:outline}
%How many chapters you have? You may have Chapter~\ref{cha:background},
%Chapter~\ref{cha:design}, Chapter~\ref{cha:methodology},
%Chapter~\ref{cha:result}, and Chapter~\ref{cha:conc}.
