\chapter{Introduction}
\label{cha:intro}
\fix{This place is why formalizing Electronic voting scheme ?}
Before I start diving deep into explaining bits and pieces of this thesis (Formal verification of Schulze Method), 
I still need to provide persuasive  argument that why did I choose to verifying Schulze algorithm in Coq theorem 
prover given the fact that Schulze is not used in any democratic election, and Coq is not serious business in 
development of mathematical theories or software artefact.   Well, the honest answer 
is that  I want to graduate (hopefully), but it's still not convincing argument because I could have chosen 
some other project and graduate.  On the serious note, this thesis started as a quest to 
find the answer of question  "Can we afford bug or bugs in software used for vote counting ?".  Given that I am 
computer scientist by profession, I would not try to justify my decisions by excessive use of philosophical 
arguments, but at this point it seems very apt to first investigate this question from philosophical point. 

"People shouldn't be afraid of their government. Governments should be afraid of their people."
― Alan Moore, V for Vendetta 


"Those who cast the vote decides nothing. Those who count the vote decide everything."
―  Joseph Stalin


"The best weapon of a dictatorship is secrecy, but the best weapon of a democracy should be the 
weapon of openness. " 
―   Niels Bohr

The answer depends on how you perceive democracy.  For a dictator, probably  bug in the 
vote software would be a natural choice, among many others,  to rig the election. If you firmly believe 
in democracy and democratic values, then 
among many other things, transparency and bug freeness in vote counting software  would also be 
in your agenda. There is no doubt that technology can play a critical role in maintaining the democratic values, 
but assuming that it is the only factor would be a gross mistake. 
In Azerbaijan's 2013 election, the running president Ilham Aliyev launched a iPhone app, to boost the 
credibility of election, which enabled the citizens of Azerbaijan to 
track the tallies as counting took place. There was just one problem. The app already showed that 
Ilham Aliyev elected before even a ballot was counted.  In this particular case, technology merely helped in 
surfacing the problem, but it did not do any other thing.  More often technology can be used to hide 
the transparency of system than making it evident specially in corrupt society for personal gain. 


Democracy is a complex system of different actors interacting with each other in certain fashion.  How to make 
these interaction more productive and better for society, I leave this to political scientists and social scientist
to figure out, and I stick to my job as a technology enabler.  This thesis is  my journey 
(with my supervisor) about finding  a way to make vote counting software more robust and transparent.




\section{Why Schulze ?}
\label{sec:thesisstatement}
Even though Schulze method is not used in any democratic election, we settled down 
with it because it is interesting  and at the same time, non-trivial. 
 Schulze's method [cite Schulze]  elects  a single winner based on 
preferential votes.  At the same time, Arrow's impossibility theorem [cite Arrow]  states that no preferential voting 
scheme can have all the desired properties established by  social choice theorist,
the Schulze's method offers a good balance. Many of these properties are already 
established in his original paper. These properties are Non-dictatorship,  Pareto,  Monotonicity, 
Resolvability, Independence of Clones 

I will discuss some of its properties in next chapter. 


A  quantitative  comparison of voting methods 
[cite An Optimal Single-Winner Preferential Voting System Based onGame Theory]  also shows that 
Schulze voting is better (in a game theoretic sense) than other, more established, systems.  The 
Schulze Method is rapidly gaining popularity in the open software community, and It is one of the most 
popular voting protocol over internet to elect candidates. At of 22 July 2019, Wikipedia entry on Schulze 
method [cite Wiki entry] shows at least 70 users, and some of 
notable users among them are Gentoo Foundation, Debian, GNU Privacy Guard (GnuPG), Ubuntu, and 
Pirate Party in Australia, Austria, Belgium, Brazil, Germany, Iceland, Italy, 
Netherlands, New Zealand, Sweden, Switzerland, and United States.  






\section{Why Coq}
\label{sec:outline}
How many chapters you have? You may have Chapter~\ref{cha:background},
Chapter~\ref{cha:design}, Chapter~\ref{cha:methodology},
Chapter~\ref{cha:result}, and Chapter~\ref{cha:conc}.
