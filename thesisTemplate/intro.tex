\chapter{Introduction to Coq}
\label{cha:intro}
\fix{Write about Hilbert's idea of mathematical formalism}

A proof assistant is a computer program which assists users in development of mathematical proofs. The idea of 
developing mathematical proofs using computer goes back to Automath (automating mathematics) project
 \citep{Geuvers2009}. The 
Automath project (1967 until the early 80's)  was initiative of De Bruijn, and the aim of the project was to develop
a language for expressing mathematical theories which can be verified by aid of computer.  Automath was first 
practical project to exploit the Curry-Howard isomorphism (proofs-as-programs and formulas-as-types)
 [reference here]. DeBruijn  was likely unaware of this correspondence, and he almost re-invented it 
 ([Wiki entry on Curry-Howard]). Automath project can be seen as the predecessor of
  proof assistants NuPrl [cite here] and Coq [cite coq].   

The Coq proof assistant [cite Coq] is interactive theorem prover based on underlying theory of Calculus of 
Inductive Construction [Cite Pual Mohring]  which itself is augmentation with inductive data-type
 of Calculus of Construction [cite Huet and Coquand].  Coq provides a highly expressive specification 
 language Gallina for development of mathematical theories and proving the theorems (specification) about these
 theories.  Even though Gallina is very expressive, writing proofs  in Gallina is very tedious and cumbersome.  
 In order to ease the proof development, Coq also provides tactics. The user interacting with Coq applies these 
 tactics to build the  Gallina term  which would otherwise be very laborious.  In this chapter, I would give a 
 brief overview of Calculus of Construction, followed by Calculus of Inductive Construction, with a example 
 of building proof directly using Gallina and show that how same proof can be build easily using tactics provided
 by Coq. In final section, I will try to justify my decision of using Coq for verifying Schulze method.
 


\section{Theoretical Foundation of Coq (CIC)}
\label{sec:thesisstatement}
The theoretical foundation of Coq is Calculus of Inductive Construction which itself is a 
augmentation with inductive data type  of Calculus of Construction (COC). Calculus of
Construction [cite Huet and Coquand] is 


\section{Introduction}
\label{sec:problemstatement}
Put your introduction here. You could use \textbackslash fix\{ABCDEFG.\} to
leave your comments, see the box at the left side. \fix{You have to rewrite your
thesis!!!} 



\section{Thesis Outline}
\label{sec:outline}
How many chapters you have? You may have Chapter~\ref{cha:background},
Chapter~\ref{cha:design}, Chapter~\ref{cha:methodology},
Chapter~\ref{cha:result}, and Chapter~\ref{cha:conc}.
