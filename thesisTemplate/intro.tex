\chapter{Introduction}
\label{cha:intro}
\fix{This place is why formalizing Electronic voting scheme ?}
Before I start diving deep into explaining bits and pieces of this thesis (Formal verification of Schulze Method), 
I still need to provide persuasive  argument that why did I choose to verifying Schulze algorithm in Coq theorem 
prover given the fact that Schulze is not used in any democratic election, and Coq is not serious business in 
development of mathematical theories or software artefact.   Well, the obvious and  honest answer 
is that  the author wants to graduate (hopefully), but it's still not convincing argument because I could have chosen 
some other project and graduate.  On the serious note, this thesis started as a quest to 
find the answer of question  "Can we afford bug or bugs in software used for vote counting ?".  Given that I am 
computer scientist by profession, I would not try to justify my decisions by excessive use of philosophical 
arguments, but at this point it seems very apt to first investigate this question from philosophical point. 

"People shouldn't be afraid of their government. Governments should be afraid of their people."
― Alan Moore, V for Vendetta 


"Those who cast the vote decides nothing. Those who count the vote decide everything."
―  Joseph Stalin


"The best weapon of a dictatorship is secrecy, but the best weapon of a democracy should be the 
weapon of openness. " 
―   Niels Bohr

The answer of the question is, it depends. If you are a dictator, then off course you 
would like the vote counting software to be full of bugs which would elect you, and if you believe 
in democracy, then you would like vote counting software to be transparent and bug free.  The author 
of this thesis believes in democracy, and  this thesis is  his journey (with his supervisor) about finding 
a way to make vote counting software more robust and transparent.


\section{Why Schulze ?}
\label{sec:thesisstatement}
Even though Schulze method is not used in any democratic election, we settled down 
with it because it is interesting and at the same time, non-trivial.  Schulze's method [cite Schulze]  elects 
a single winner based on 
preferential votes.  However; Arrow's impossibility theorem [cite Arrow]  states that no preferential voting 
scheme can have all the desired properties established by  social choice theorist,
the Schulze's method offers a good balance of properties. Many of these properties are already 
established in his original paper.



It's one of the most popular voting protocol 
over internet to elect candidates. 






\section{Why Coq}
\label{sec:outline}
How many chapters you have? You may have Chapter~\ref{cha:background},
Chapter~\ref{cha:design}, Chapter~\ref{cha:methodology},
Chapter~\ref{cha:result}, and Chapter~\ref{cha:conc}.
