\chapter*{Abstract}
\setlength{\parindent}{2em}
\setlength{\parskip}{1em}
\addcontentsline{toc}{chapter}{Abstract}

%\vspace{-1em}


Since the introduction of secret ballot by Australians in 1855, 
paper (ballots) are widely used around the world to record 
the preferences of eligible voters. Paper ballots provide three 
important ingredient: correctness, privacy, and verifiability. 
However, the paper ballot election poses various  other challenges, e.g. 
slow for large democracies like India,  error prone for complex voting method 
like single transferable vote, and poses operational challenges for 
massive countries like Australia. In order to solve these problems and various others, 
many countries are adopting electronic voting. However, 
electronic voting has a whole new set of problems. In most cases, the software 
program used to conduct the election has numerous problems, including, but no limited to, 
counting bugs, ballot identification, etc. Moreover, 
these software programs are treated as commercial in confidence and 
are not allowed to be inspected by general member of public. 
As a consequence, the result produced by these software programs 
can not be substantiated.

In this thesis, we address the three main concerns posed by electronic voting, i.e. 
correctness, privacy, and verifiability. We address the correctness concern by using 
theorem prover to implement the vote counting algorithm, 
privacy concern by means using homomorphic encryption, and verifiability concern 
by means of generating a independently checkable certificate.  Our work 
has been carried out in Coq theorem prover.

%%% Local Variables: 
%%% mode: latex
%%% TeX-master: "paper"
%%% End: 