\chapter*{Abstract}
\addcontentsline{toc}{chapter}{Abstract}
\vspace{-1em}


Since the introduction of secret ballot by Australians in 1855, 
paper (ballots) are widely used around the world to record 
the preferences of eligible voters.
The reason is that they provide two important ingredient: privacy and verifiability. 
However, the paper ballot election poses various challenges, e.g.  
slow for large democracies like India, error prone for complex voting method like 
single transferable vote, and operational challenge for big countries like Australia.  
Moreover,  they are environmental hazard because it produces 
a lot paper waste.  In order to solve these problems and various other, 
many countries are introducing electronic voting. But electronic voting 
has a whole new set of problems. In most cases, the software program used 
to conduct the election has numerous problems, including, but no limited to, 
counting bugs, hard coded password, ballot identification, etc. Moreover, 
these software programs are treated as commercial in confidence and 
are not allowed to be inspected by general member of public. 
As a consequence, the result produced these software programs 
can not be substantiated. 

In this thesis, we address the three main concerns, privacy, verifiability, 
and correctness, posed by electronic voting. We show that privacy can be achieved 
by  

This thesis builds on and contributes to work in the areas of electronic voting. 
The original contribution of my thesis is achieving correctness, privacy and verifiability in 
electronic voting. Although studies in electronic voting has examined verifiability and correctness, 
there has not been privacy.   As such, this study provides additional insight into correctness with 
emphasis on verifiability by generating a proof certificate (additional piece of data for auditing) . 
This study is important to my discipline because it bridges the gap between electronic voting 
and paper ballot election. But what excites me the most about this study is making the democracy accessible to everyone
and making there voice heard. 

%%% Local Variables: 
%%% mode: latex
%%% TeX-master: "paper"
%%% End: 