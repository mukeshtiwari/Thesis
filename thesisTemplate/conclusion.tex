\chapter{Conclusion and Future Work}
\label{cha:conc}

\epigraph{Education is the most powerful weapon which you can use to change the world.} 
{\textit{Nelson Mandela}}

This chapter is summarizes the key outcomes of this dissertation, followed by possible 
future work. 

\section{Conclusion}
Recall that the journey started with the purpose to make the electronic voting 
process transparent and trustworthy.  The premise was:

\begin{displayquote}
Given the potential advantages of electronic voting, we need to address correctness,
privacy, and verifiability concerns for its widespread adoption.
\end{displayquote}

The current state of 
art software program used by many governments is mostly black-box which 
takes a pile of ballots and produces result.  To improve the current situation, 
we focussed on answering the 
four main concerns: \textit{(i) correctness}, \textit{(ii) privacy},
\textit{(iii) coercion resistance}, and \textit{(iv) verifiability (tallied-as-cast)} by 
using \textit{Schulze method} an example. 

\subsection{Correctness}
With the intentions to solve the 
correctness in electronic voting, we approached from 
mathematical logic route.  Rather than implementing the Schulze method, 
we gave a logical specification of the Schulze winning condition and losing 
condition with respected to already computed margin function. 
These specifications were simple enough that 
one can inspect them, and 
make sure that the intent is captured correctly in these specifications. 
Moreover, we proved the correctness properties about our specification.
Also, we put forward the idea of formalizing the properties 
of voting protocol, in our case Schulze method, in the framework developed by Kenneth Arrow 
in theorem prover itself. For example, 
in the last chapter, we proved the \textit{Condorcet} winner property 
of Schulze method. Given that ballot counting is one of the most 
crucial phase of any election, the correctness of counting software 
should be explored from all the possible directions. 

\subsection{Verifiability}
We answered the verifiability issue by producing a 
independently checkable scrutiny sheet. In our case, 
it contained the step by step computation of margin, 
winners and loser with the proof why they win or 
lose the election. This scrutiny 
sheet can be used any independent third to verify the 
outcome of election. In case of plain-text ballots, 
achieving verifiability was trivial, but the encrypted 
ballot case was very complex. The reason 
for complexity was the inherent nature of cryptography, 
so to keep the encrypted ballot election verifiable 
we augmented the scrutiny sheet with zero-knowledge-proofs. 
Finally, we developed a formally verified certificate checker to ease the 
auditing of election (although, we ended up developing 
checker of different election). 

\subsection{Privacy and Coercion Resistance}
Our approach to privacy and coercion problem was homomorphic encryption. 
To keep the content of ballot private, we did not decrypt
any individual ballot and computed the final tally homomorphically 
by multiplying the cipher-texts. In the final step, we decrypted 
the fully computed tally, which 
in turn did not reveal any individual ballots.  Since there 
was no decryption of any individual ballot, there was 
no way a voter could have convinced any one about her 
choices. 


\section{Future Work}
\label{sec:future}
\subsection{Formalizing Cryptographic Entities}
During the formalization, we assume the cryptographic primitives 
for various construction, e.g. encryption primitive, decryption primitive,
zero-knowledge-proof primitive, etc. Moreover,  
we assumed axioms about their correctness property. 
Formalizing all these primitives and proving the 
axioms we assumed would further close the trust 
gap.  


\subsection{Formalizing Properties of Schulze Method}
We have formalized just two properties, 
\textit{Condorcet winner}, and \textit{Reversal symmetry},
but taking it further and proving all the properties 
would put more trust in the implementation. Moreover, 
it would be a good stress testing for the specification/implementation 
and see how far it can go. 

\subsection{Formally Verified Checker}
Another interesting avenue would be to explore
the formally verified certificate checker.  In our formally verified 
certificate checker, we extracted OCaml code for certificate checking. 
However, we wrote a substantial amount of unverified OCaml code 
for parsing the scrutiny sheet. To alleviate this kind of concerns, 
it would be worth exploring a verified parser, and, probably, 
evaluating the whole certificate inside Coq environment. 
There has already been verified parsers written in 
Coq \citep{10.1007/978-3-642-28869-2_20}, 
and given that we proved the two large primes inside a Coq, 
it does not seem a far fetched concept. 

\subsection{Risk Limiting Audit for Preferential Voting Scheme}
One challenging potential opportunity would be developing 
a risk limiting audit system for Schulze method. In nutshell, 
risk limiting audit is a method to audit the election by 
randomly sampling the ballots. Risk limiting is very well 
understood in the first-past-the-post voting system, 
and some of the work has been done in the context 
IRV elections \citep{10.1007/978-3-030-00419-4_2},
but so far none, to the best of my knowledge, 
for Schulze method.

\subsection{Formalizing Code Extraction}
This is orthogonal, but very important from the perspective of 
electronic voting.  One of the biggest concern with code extraction of Coq in OCaml/Haskell 
is that all the security properties proved inside Coq are no longer valid 
in OCaml/Haskell. Taking the mission critical importance of electronic voting into account, 
we need a mechanism to translate the proofs all the way up to assembly level. 
CertiCoq \citep{anand2017certicoq} seems promising, but it is not very mature yet. The other possibility is 
using the CakeML \citep{Kumar:2014:CVI} to develop the electronic voting  
schemes. 




