\chapter{Background}
\label{cha:background}
\setlength{\parindent}{2em}
\setlength{\parskip}{1em}

\epigraph{People shouldn't be afraid of their government. Governments should be afraid of their people.} 
{\textit{Alan Moore, V for Vendetta }}
  Counting ballots by hand is a tedious, error prone, slow, and costly process. 
  For example, the Senate election conducted in Western Australia in September 2013 was 
  declared void by the high court because of the loss of 1370 votes. It was 
  re-conducted in April 2014 at the cost of 20 Million 
  AUD with additional  delay in results \citep{Aussentate}. Before introduction to 
  electronic voting machines in India, it used to take months to declare the result.
  As a consequence, many countries 
  are now adopting electronic voting to alleviate the problems introduced by hand counting. 
  The world can be divided into five broad categories according to 
  the usage of electronic voting \citep{Evoting} (Figure \ref{fig:world_electronic_voting_map}): i) No electronic 
  voting (Grey Area), ii)
  Discussion and/or voting technology pilots (Yellow Area), 
  iii) Discussion and concrete plans for Internet voting (Orange Area),
  iv) Ballot scanners, Electronic Voting Machines, and Internet Voting (Green and Dark Green),
  v) Withdrawn voting technology because of public concern (Red Area).
    \begin{figure}[!htb]
	\begin{center}
	\includegraphics[width=\textwidth,height=\textheight,keepaspectratio]{e-voting_worldmap_2015.pdf}
	\caption{World map of Electronic Voting (source: \url{https://www.e-voting.cc/en/it-elections/world-map/})}
	\label{fig:world_electronic_voting_map}
	\end{center}
  \end{figure}  
 
 
\textbf{Chapter Outline:}
 In the Section \ref{secback:electronic_voting}, we discuss 
 the major concern in electronic voting, bugs in the software/hardware, 
 by high lighting the 
 state of electronic voting in Australia, Germany, India, 
 and Netherlands. In the following Section \ref{secback:correctness_formal}, we  give 
 two anecdotes that how formal verification helped in achieving 
 the correctness and eliminating bug in \textit{CompCert} \citep{DBLP:conf/popl/Leroy06},  a formally verified C compiler, 
 and \textit{Athelon}, a microprocessor designed by Advanced Micro Devices (AMD), with 
 the emphasis that we should formally verify the software programs
 used in electronic voting to alleviate the concerns of 
 software/hardware bug. Section \ref{secback:varifiability} 
 emphasizes that formal verification is not enough
 to establish the trust and puts forward the concept of 
 scrutiny sheet (\ref{secback:scruntiny_sheet}), which 
 can be used independently to attest the result of an election, 
 to achieve verifiability. Finally, we summarize in the Section \ref{secback:summary_back}
 by emphasizing that formal verification and verifiability both are needed 
 to ensure the trust in electronic voting. 
 
  

\section{Electronic Voting}
 \label{secback:electronic_voting}
  Electronic voting is projected as a step towards the future with 
  many benefits, such as increased voter turnout, faster result, 
  accessible to everyone including challenged voters, and reduced 
  carbon footprint (for each 
  national election, India saves about 10,000 tonnes of the ballot 
  paper by using electronic voting machines). 
  There is no doubt that electronic voting has many advantages 
  over paper ballots, but it is certainly not flawless.  
  Electronic voting makes 
  the process faster, but it has its own layer of added complexities 
  which creates trust issues amongst voters. For this reason, some countries 
  who were the early adopters were also the early abandoner, e.g.
  Germany, and The Netherlands (countries in red color in the world map 
  \ref{fig:world_electronic_voting_map}).
  
  
  \textbf{Germany:} In 2005 German election, two voters filed a case in the German 
  Constitutional Court (Bundesverfassungsgericht) because their 
  appeal to scrutinize the election 
  was not heeded by the Committee. They argued that using electronic 
  voting machines to conduct the election was unconstitutional. Furthermore,
  they added that
  these machines could be hacked, hence results of the 2005 election 
  could not be trusted on the grounds 
  of public examinability of elections according to German Constitution 
  (Basic Law for the Federal Republic of Germany) \citep{Germanconst}. 
  The Court noted that, under the constitution, elections are 
  required to be public in nature \citep{Germanconst}:
  
  \begin{displayquote}
  The principle of the public nature of elections requires that all 
  essential steps in the elections are subject to public examinability
  unless other constitutional interests justify an exception. 
  Particular significance attaches here to the monitoring of the 
  election act and to the ascertainment of the election result.
  \end{displayquote}
  
 
 
  \noindent	
  In its verdict, the court did not rule out or prevent the usage of electronic 
  voting machines,  but suggested to make the process more 
  transparent and trustworthy \citep{Germanconst}:
  
  \begin{displayquote}
  The legislature is not prevented from using electronic voting machines 
  in the elections if the constitutionally required possibility of a 
  reliable correctness check is ensured. In particular, voting machines 
  are conceivable in which the votes are recorded elsewhere in addition
   to electronic storage. This is for instance possible with electronic
   voting machines which print out a visible paper report of the vote 
   cast for the respective voter, in addition to electronic recording 
   of the vote, which can be checked prior to the final ballot and is
    then collected to facilitate subsequent checking. Monitoring that is
     independent of the electronic vote record also remains possible when
     systems are deployed in which the voter marks a voting slip and the 
     election decision is recorded simultaneously, 
     or subsequently by electronic means in 
     order to evaluate these by electronic means at the end of the 
     election day.
     \end{displayquote}
  
  \textbf{The Netherlands:}
  The Netherlands was among a few countries who adopted electronic voting 
  in the early nineties (1990), but it did not go very well in the long 
  run and was abolished in 2008 \citep{Jacobs2009}. 
  The reason for abolishing the electronic voting was that   
  the voting machines used in elections were susceptible to many attacks,
  and the results of elections conducted using these machines 
  were not publicly verifiable.  Besides, a Dutch public foundation, Wij vertrouwen stemcomputers niet
  (We do not trust voting computers), demonstrated that the e-voting machines 
  used in the election leaks enough information to guess the choice of a voter 
  at a distance of 20 to 30 meters from 
  the polling booths  \citep{NetherlandsYouTube}.
  
  
  
  Germany and The Netherlands are some of the rare cases where 
  electronic voting was withdrawn because it was not able to 
  replicate the same trust environment as created by paper 
  ballot systems whereas Australia, and India continued 
  with electronic voting despite having the concerns expressed 
  by researchers about the security of system. 
  
  \textbf{India:}
  India, one of the largest democracies in world, 
  uses electronic voting machines (also known as EVMs) for national 
  and state level  elections despite the fact that many political parties have raised security 
  concern against them. Moreover, it has already been shown in \citep{Wolchok:2010:SAI:1866307.1866309}  that it 
  is possible to manipulate the election results. In their attack, they replaced the parts 
  of electronic voting machine  with malicious look alike components. These components 
  were capable of receiving instruction over wireless communication. As a result, any malicious 
  attacker can control these components from nearby vicinity by sending 
  instructions over wireless channel by using a simple hand-held device and 
  can  manipulate the results in their favour \footnote{https://indiaevm.org/}.
  India is mainly criticised for keeping the design of electronic voting machines 
  a closely guard secrets (security by obscurity). However, it is not impossible to get access of 
  these machines as shown by \citep{Wolchok:2010:SAI:1866307.1866309}. 
  The worst part, the design of these machines were never audited by 
  any independent third party. 
  
  

 \textbf{Australia:}
  In March, 2015 state election 
  of New South Wales, Australia, a Internet voting system, iVote,    
  was used and 280,000 votes were cast through it. NSW Election 
  commissioner claimed that it was:
  
  \begin{displayquote} 
  It's fully encrypted and safeguarded, it can't be tampered with, 
 and for the first time people can actually after they've voted 
 go into the system and check to see how they voted just to make 
 sure everything was as they intended \citep{NSWelection}.
 \end{displayquote}
 

  \noindent
  The voting on iVote 
  opened on Monday March 16 and continued until March 28. On 22 March,
  two security researchers, Vanessa Teague and J. Alex Halderman, 
  announced that iVote has critical security bug, and they demonstrated 
  that it was good enough to steal any ballot. From their paper
  \citep{10.1007/978-3-319-22270-7_3}:
  
  \begin{displayquote}
  
  
   While the election was going on, we performed an independent,
   uninvited security analysis of public portions of the iVote 
   system. We discovered critical security flaws that would allow
   a network-based attacker to perform downgrade-to-export 
   attacks, defeat TLS, and inject malicious code 
   into browsers during voting. We showed that an attacker could
   exploit these flaws to violate ballot privacy and steal votes. 
   We also identified several methods by which an attacker could
   defeat the verification mechanisms built into the iVote design.
   
   \end{displayquote}
  
  \noindent
  Basically, New South Wales ran a online election for 6 days on 
  buggy software which was susceptible to many attacks with a possible 
  outcome of tampered ballot without anyone noticing it. 

   We do not have to think very hard to figure out the reasons for 
   these debacles. 
   There are various factors for these debacles, but one of the most 
   common denominator among all these debacles,
   which contributed significantly,  is the software/hardware used in the election process 
   had numerous  bugs. 
   But, this begs the question: why various governments 
   were (Germany, Netherlands)/are (Australia, India) using such poor quality software programs
   in the first place for conducting elections? 
   In general, no entity related to government or electoral commission develops the software programs 
   for electronic voting, and predominantly it is outsourced 
   to companies having experience in electronic voting software development.   
   Most of these companies produce poor quality software because 
   of unrealistic schedule, lack of proper software testing practices, 
   lack of technical knowledge, etc. In the report, \citep{TSwiss} stated 
   in the source of the problem of SwissPost debacle as:
   
   \begin{displayquote}
    Nothing in our analysis suggests that this problem was introduced deliberately.  
    It is entirely consistent with a naive implementation of a complex cryptographic
    protocol by well-intentioned people who lacked a full understanding of its security
    assumptions and other important details.  Of course,  if someone did want to 
    introduce an opportunity for manipulation, the best method would be one that 
    could be explained away as an accident if it was found.  
    We simply do not see any evidence either way.
    \end{displayquote}
   
   \noindent 
   Moreover, more often than not, these software programs
   are closely guarded secrets and their source 
   code in not open for public scrutiny because of commercial 
   interests of companies \citep{AEC:2013:LMM} involved in the 
   process.  Overall, the whole process lacks transparency, which 
   violates the fundamental principals of democracy, i.e. openness. 
  
   The process  of turning a  idea into a 
   concrete software, also known as software development process, involves 
   requirement gathering, software design, implementation, testing, and maintenance. 
   During this whole process of software development, there are various factors 
   which affects the quality of software.  However, 
   throughout this entire software development (and maintenance life) cycle,
   software testing is the only mechanism for quality assurance, but
   it is not enough for 
   instilling the confidence in software that it is bug free 
   as stated by Edsger W. Dijkstra \citep{Dijkstra:1972:HP:355604.361591}:
   \begin{displayquote}
   Program testing can be used to show the presence of bugs, 
    but never to show their absence!
   \end{displayquote}
  
   
   In the next section, given the mission critical importance of electronic voting software, 
   we will discuss that software testing 
   is not sufficient to achieve the software trustworthiness \citep{10.1007/978-3-642-35795-4_5},  and we should 
   prove the  correctness of these software
   by using formal verification techniques \citep{BECKERT2014115}.
   Furthermore, we will argue that  having a formal verification software development methodology \citep{10.1007/978-3-319-95582-7_38}
    would alleviate 
	the bug problem with two case studies, \textit{CompCert} \citep{DBLP:conf/popl/Leroy06} and \textit{Athelon}, as a supporting evidence. The success of these 
	case studies should be a good motivation for us to adopt formal method for electronic voting software development. 
   
   
   \section{Correctness: Formal Method Approach}
   \label{secback:correctness_formal}

	Formal verification has been successfully applied in many areas, and 
	some of the notable software programs are 
	verified C compiler CompCert \citep{DBLP:conf/popl/Leroy06}, verified ML compiler
	CakeML \citep{Kumar:2014:CVI}, verified LLVM 
	Vellvm \citep{10.1007/978-3-642-35308-6_6}, verified cryptography 
	Fiat-crypto \citep{DBLP:conf/sp/ErbsenPGSC19}, verified 
	operating system CertiKOS \citep{DBLP:conf/apsys/GuVFSC11} and 
	SeL4 \citep{DBLP:conf/sosp/KleinEHACDEEKNSTW09}, verified theorem 
	prover Milwa \citep{DBLP:conf/itp/MyreenD14}, 
	verified crash resistant file system FSCQ \citep{Chen:2015:UCH:2815400.2815402}, 
	verified distributed system 
	Verdi \citep{DBLP:conf/pldi/WilcoxWPTWEA15}, mechanisation of 
	Four Color Theorem \citep{10.1007/978-3-540-87827-8_28}, 
	Fundamental Theorem of 
	Algebra \citep{10.1007/3-540-45842-5_7}, and Kepler Conjecture \citep{hales-kepler}.
	None of these are toy projects, and it has taken years 
	to develop and verify them. Also, some of these products 
	are used commercially,  e.g CompCert is used by the AIRBUS and the MTU (Motoren und Turbinen Union) 
	Friedrichshafen\footnote{https://www.absint.com/compcert/},  Fiat-crypto is used 
	in the Google's BoringSSL library for elliptic-curve 
	arithmetic\footnote{https://deepspec.org/entry/Project/Cryptography}. 
	Based on the cost and efforts of these project, 
	the very basic question to ponder about using formal method to develop 
	software:  does formal verification produce bug free software?
	
	
	We give two anecdotes to answer this question. 
	One of the most basic way to 
	break the software is generating random tests and throwing it to 
	the software under consideration \citep{Miller:1990:ESR:96267.96279}.
	\citep{Yang:2011:FUB:1993316.1993532} developed random 
	C program generator and used these programs to test various 
	compilers. In three years of its usage, they have found 325 unknown
	bugs in various compiler including 
	GCC\footnote{https://embed.cs.utah.edu/csmith/gcc-bugs.html} and 
	LLVM\footnote{https://embed.cs.utah.edu/csmith/llvm-bugs.html}; however, 
	they could not find any bug in verified component of CompCert. 
	In their own words \citep{Yang:2011:FUB:1993316.1993532}:
	
	\begin{displayquote}
	
	The striking thing about our CompCert results is that the middle-end 
	bugs we found in all other compilers are absent. As of early 2011,
	the under-development version of CompCert is the only compiler we
	have tested for which Csmith cannot find wrong-code errors. This is
	not for lack of trying: we have devoted about six CPU-years to the
	task. The apparent unbreakability of CompCert supports a strong
	argument that developing compiler optimizations within a proof
	framework, where safety checks are explicit and machine-checked,
	has tangible benefits for compiler users.
	
	\end{displayquote}
	
	\noindent
	Formal verification is not only helpful in proving the correctness, 
	but sometimes, it helps in uncovering the bugs in design of
	software. ACL2, a Lisp based theorem prover, helped 
	AMD to uncover a floating point bug in Athelon processor which 
	has survived 80 million floating point test cases! 
	In the paper, Milestones from the Pure Lisp theorem prover to ACL2
	\citep{Moore2019}, Moore, one of the developer of ACL2, writes:
	
	\begin{displayquote}
	
	When AMD developed their translator 
	from their register-transfer language (in which designs
	are expressed) to ACL2 functions they ran 80 million 
	floating point test cases through the ACL2 model of 
	Athlon’s FMUL and their own RTL simulator. However, the 
	subsequent proof attempt exposed bugs not covered by the
	test suite. These bugs were fixed before the Athlon was 
	fabricated.
	
	\end{displayquote}
  
   
   There are numerous instances where formal verification 
	was very useful, and it caught the lurking bugs in design in early 
	stage which could never have been found by testing.
	For electronic voting software used in democratic election, where we 
	can not afford to lose a single ballot or miscalculation 
	or any undefined behaviour, should be developed 
	using formal method techniques.
	In order to ascertain that the formal verification of voting software has been 
	carried out diligently, one therefore needs to
	\begin{enumerate}
	\item read, understand and validate the formal specification: is it 
	error free, and does it indeed reflect the intended functionality?
	\item scrutinize the formal correctness proof: has the verification
	been carried out with due diligence, is the proof complete or does
	it rely on other assumptions?

	\end{enumerate}			
	
	\noindent
	The above mentioned requirements can be met by publishing or open sourcing
	both the
	specification and the correctness proof so that the specification
	can be analysed, and the proof can be replayed (inside the tool used for 
	verification) by any independent third party. Both need a
	considerable amount of expertise, but it can be carried out by (ideally
	more than one group of) domain experts.
	

	
	  
	 
		
 \section{Verifiability: Trust in Electronic Voting}
 \label{secback:varifiability}
 Given that elections are  the cornerstone of our democracy, 
   electronic voting software programs should be considered as mission-critical systems, 
   and therefore they should be developed with highest possible rigour. 
  Formal verification is useful in producing the 
   bug free code, but we solely can not 
   establish the trust in the system based on argument of 
   formal verification. The reason is that how would a voter:
   \begin{itemize}
   \item ascertain that it was indeed the verified program that was
	executed in order to obtain the claimed results?
	
   \item ensure that the computing equipment on which the (verified)
	program is executed has not been tampered with or is otherwise
	compromised?
	
   \end{itemize}
   
   \noindent
   
    Recall that the notion of (end-to-end) verifiability in electronic voting is
 ascertaining the outcome of an election without trusting any 
 machine involved in the process. 
   In general, formal verification is certainly a necessary thing 
   in developing the software programs for electronic voting, 
   but it is not sufficient because it does not provide verifiability.
   Combing both \emph{verification} of the
	software program that counts votes, and
	\emph{verifiability} of individual counts are critical for
	building trust in an election process. These two facts, 
	 \emph{verification}  and \emph{verifiability}, 
	 can also be viewed as the
    two sides of a coin from the perspective of two major stake holders of 
    a democracy: i) electoral commission/government, and, ii) voters/participants.  
    Using a formally 
    verified software program to count the ballots would increase the confidence 
    of an electoral commission that it has announced and published the correct result. 
    Moreover,  the published result always verifies.  Therefore,  it would increase the confidence
    of voters in the system. 
	 
	 
	Given the mission-critical importance of
	 correctness of vote-counting,
	both for the legal integrity of the process and for
	building public trust,  it is crucial to replace the
	currently used black-box software for vote-counting with a
	counterpart that is both verified and produces 
	evidence which can later be used to certify
	the outcome of election \citep{Bernhard:2017:PES} \citep{Rivest:2008:PTRS}.
	  		
		

  
	       
    
   \subsection{Scrutiny Sheet}
   \label{secback:scruntiny_sheet}
   A scrutiny sheet is the tabulation of relevant data to verify the result of election. 
   The idea of requiring that computations provide not only results, but also a witness attesting
    to the correctness of the computation is not new
	and has been put forward in \citep{89397}
	\citep{MCCONNELL2011119}
	\citep{Arkoudas:2005:DRC}, and, in the context of electronic voting by  \citep{Schurmann:2009:EET} \citep{Pattinson:2015:VCM}.
	 In general, the idea of computation is that a computable function \textit{f} takes 
	an input \textit{x} and produces output \textit{y} (figure \ref{fig:fnxy}); however, in case of
	 certified computation, 
	the computable function \textit{f} on the given input \textit{x}, not only produces the output \textit{y},
	but it also produces a witness \textit{w} for the fact that $ f (x) = y$ (Figure \ref{fig:fnxyw}).
	
	\begin{figure}[!h]
	\centering
  \includegraphics[width=0.5\linewidth]{figs/function_fx.png}
  \caption{Function f computing y on input x}
  \label{fig:fnxy}
  \end{figure} 
  
  \begin{figure}[!h]
  \centering
  \includegraphics[width=0.5\linewidth]{figs/funcxy.png}
  \caption{Function f computing y and producing witness w on input x}
  \label{fig:fnxyw}
  \end{figure} 


   \noindent 
    As a simple example, below is a certificate which has been generated by a program which computes the greatest common divisor 
    of two numbers. The program has produced the certificate (piece of data)
    on the concrete input 34 and 21:
    \begin{center}
  
   \begin{verbatim}
     gcd 34 21
     -------------
     gcd 21 13
     -------------
          .
     -------------
     gcd 1 0 
     -------------
        1
   \end{verbatim}
    \end{center}
   
   
   In order to verify the correctness of the computation of greatest common divisor, we need to 
    make sure that one of the rules of Euclidean algorithm, given below, is applicable at 
    each line of the certificate:
   \begin{enumerate}
   \item Rule-zero: $\forall$ x, $\gcd$ x 0 = x
   \item Rule-inductive: $\forall$ x y, $\gcd$ x y = $\gcd$ y (mod x y)
   \end{enumerate}

 \noindent	
  The same certificate (augmented with rules and variables instantiated) from certificate-checker perspective.
    \begin{minted}[fontsize=\footnotesize]{coq}
     gcd 34 21 = gcd 21 13 Rule-inductive: x := 34, y := 21, mod 34 21 = 13
     ---------------------------------------------------------------------
     gcd 21 13  = gcd 13 8 Rule-inductive: x := 21, y := 13, mod 21 13 = 8
     ---------------------------------------------------------------------
          .
     ---------------------------------------------------------------------
     gcd 1 0 = 1   Rule-zero: x := 0 
   \end{minted}
   
   
   Unfortunately, the example we gave,  greatest common divisor, is very simple and not very
   helpful to put forward the usefulness of certificate in perspective.
   However, this approach, generating a certificate to certify the computation later, is very useful in context 
   of complicated and unverified  programs.   One such example is algorithmic library  
   LEDA \citep{Mehlhorn:1995:LPC:204865.204889} ("Library of Efficient Data types and Algorithms") 
   written in C++.  Checkers are an integral part of the LEDA 
   which can later be invoked by user to certify that the result produced by unverified 
   code is correct. 
   Initially, the checkers came with library were unverified, and 
   Kurt Mehlhorn defended this decision by admitting that \citep{Alkassar2014}: 
   \begin{displayquote}
   Checkers are simple programs with little algorithmic 
   complexity. Hence, one may assume that their implementations are correct.
	\end{displayquote}   
	\noindent
    Later, the 
   checkers \citep{Alkassar2014} were verified by using VCC \citep{10.1007/978-3-642-03359-9_2}, 
   and Isabelle/HOL \citep{Nipkow:2002:IPA:1791547}.  One advantages of this approach is 
   that it is easier to formally verify the checker than the algorithm itself, because 
   checkers are very simple in nature, and this approach scales very well. 
   
   One may ask the question that can we follow the same approach for vote counting,
   i.e. unverified counting code, and verified checker?  The answer is: it depends. 
   If the verified checker validates the result, i.e. it returns true on  
   the certificate generated by a unverified 
   vote counting program then everything is fine from every stakeholders' perspective.
   However, what about the situation when the verified checker invalidates the result, i.e. 
   it returns false on the certificate generated by the unverified vote counting program? 
   This kind of situation must be dealt carefully by an electoral commission, and the commission should
   inspect everything carefully including the vote counting software and various other thing involved in the
   process.
   To eliminate this kind of problematic situation, we propose: i) a formally verified vote counting
   software which produces the result with evidence (certificate), and ii) formally verified 
   certificate checker. The advantage of this approach is that the result produced is always correct (modulo 
   specification), a confidence building measure for electoral commission. Furthermore, the verified checker would 
   always return true on the evidence (certificate) produced by verified vote counting program, confidence booster for 
   voters into the deployed system.  
  

  
  
\section{Privacy}
   So far we have argued that given the  importance of vote counting software in an election, we should 
   develop it with highest possible rigour, i.e. formal verification and open source it for public scrutiny. Furthermore, 
   to increase the public trust in vote counting process, we should strive for verifiability by tabulating all the 
   relevant data (scrutiny sheet) on a public bulletin board.  Many jurisdictions follow some derived version of these practises, e.g.  ACT Electoral 
   Commission\footnote{\url{https://www.elections.act.gov.au/elections_and_voting/electronic_voting_and_counting}}
   has published all the ballots and vote counting software since 2001, the year electronic voting  was introduced 
   (surprisingly,  this year,  2020,  ACT Electoral Commission did not  publish the vote counting software, a step back than 
   going forward).  However, for preferential voting schemes publishing all the ballots in plaintext could lead to ballot identification
   via "Italian" attack \citep{Otten}, which we have already discussed in the previous chapter.    This may appear surprising in the 
   first sight,  but if we analyse it more carefully,  we can see that verifiability requirement is causing this privacy issue.   Indeed, 
   privacy and verifiability are conflicting requirements \citep{JONKER20131}.  
   On the one hand,  (end-to-end) verifiability, in a broad sense,  is about making sure that
   every voter can see that her ballot is included in the final tally and the final tally is produced based on all the published ballots, 
   while on the other, privacy is about not letting any one link a ballot to a particular voter,  even if that particular voter wants.  
   The only way to resolve the tension between privacy 
   and verifiability is to use various cryptographic primitives,  which hide the data  by using encryption (privacy) and 
   ensure that every claim is accompanied by a mathematical proofs (verifiability) (these proofs are in form of data which 
   can be checked by anyone,  and one such proof system is zero-knowledge-proof).  
   
   
   The "Italian" attack may seem far-fetched to the reader; however,  a political scientist,  Dr. Kevin Bonham,
    was able to link  15 ballots, when he was studying the Tasmania Senate election,  
   to its voters by looking at the preferences \citep{TasmaniaVoting}. The preferences on these 15 ballots were so rare, amongst 
   all the cast ballots,  that  he conjectured that these votes could be a recommendation from some lobby group 
   or some technological issue. 
   Later, he found the voters of these 15 ballots on a private Facebook 
   group where one person came up with a rough order,  which eventually led to this particular order after 
   some refinement.  He pointed \citep{TasmaniaVoting} towards a potential privacy and coercions issue:
   
   \begin{displayquote}
   
   There is a minor privacy implication of the increased uniqueness of Senate votes under the new system.  In theory, coercion to 
   vote a given way is useless in the Australian context outside of postal voting, because the ballot box is secret, so the coercer 
   cannot establish how the person voted.  However, in the Senate it is not that difficult to establish whether or not anyone at all 
   voted in a particular manner, so it is also often possible to confirm that there is a specific way that nobody at all (either in general or 
   at least at a specific booth) voted.   So in theory a coercive person could direct a voter to vote in a certain way, and then check the 
   files to verify whether anyone in a booth, or at all, had voted in that way. The coercer could even choose to "sign" the directed vote 
   with an unlikely combination of candidates or an unlikely set of repetitions, not affecting the fate of the vote, to make it more likely 
   that the coerced vote would be unique in the whole state and not just at booth level.  That way if the vote did not appear, they 
   would know the voter had not voted as they directed.
   
   \end{displayquote}
   
   
   We champion,  considering the importance of privacy in elections,  that cryptographic methods should be 
   used to hide (encrypt) the preferences in the ballot of a given voter.  
   Moreover, every claim about the (encrypted) ballot
   should be accompanied by mathematical proofs (in form of data) so that the claims can be verified 
   by any  independent third party, including the voter itself.  At this point,  we would like to point that 
   cryptography makes the auditing difficult for general public because of the complex mathematics 
   involved in the process.  We believe nonetheless that there would enough voters and independent 
   auditors, having the knowledge of required cryptographic schemes, who would be able to audit the election 
   and ascertain the outcome. 
   
   
\section{Summary}
\label{secback:summary_back}

In this chapter, we argued that to make the electronic voting  more trustworthy 
and privacy preserving, we need three ingredient: 
\begin{enumerate}
   \item correctness: formal verification of vote counting software
   \item verifiability: tabulation of all the relevant data on a public bulletin board
   \item privacy: hiding the options of a voter by means of cryptography 
\end{enumerate}
\noindent 
The system,  by combing these three concepts,  would be a formally 
verified (encrypted) vote counting software that not only computes the 
final result but additionally produces an independently verifiable certificate, 
which attests to the correctness of the computation.  The major advantage of 
a certificate-producing formally verified vote-counting program,  all the external
parties or stakeholders can satisfy themselves to the
correctness of the count by checking the certificate.  
Moreover, we also argued 
that including a formally verified certificate checker would boost the confidence 
of both,  electoral commission and voters. 
  
In the next chapter, we will  briefly discuss about the Coq theorem prover and basic cryptographic primitives which 
would enable us later in counting encrypted ballots (without revealing any information about the
voters' choice).